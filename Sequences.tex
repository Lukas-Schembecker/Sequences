\documentclass{article}
\usepackage[english]{babel}
\usepackage{enumerate, latexsym, amssymb, amsmath}
\usepackage{framed, multicol}
\newenvironment{forthel}{\begin{leftbar}}{\end{leftbar}}

%%%%%%%%%% Start TeXmacs macros
\newcommand{\tmaffiliation}[1]{\\ #1}
\newcommand{\tmem}[1]{{\em #1\/}}
\newenvironment{enumeratenumeric}{\begin{enumerate}[1.] }{\end{enumerate}}
\newenvironment{proof}{\noindent\textbf{Proof\ }}{\hspace*{\fill}$\Box$\medskip}
\newenvironment{quoteenv}{\begin{quote} }{\end{quote}}
\newenvironment{subproof}{\begin{list}{}{}
		\item[\text{Proof}]}{\hfill $\Box$ \end{list}}
\newtheorem{axiom}{Axiom}
\newtheorem{lemma}{Lemma}
\newtheorem{theorem}{Theorem}
\newtheorem{definition}{Definition}
\newtheorem{signature}{Signature}
\newtheorem{proposition}{Proposition}
%%%%%%%%%% End TeXmacs macros

\newcommand{\event}{UITP 2018}
\newcommand{\dom}{Dom}
\newcommand{\fun}{aFunction}
\newcommand{\sym}{sym}
\newcommand{\halfline}{{\vspace{3pt}}}
\newcommand{\tab}{{\hspace{1cm}}}
\newcommand{\ball}[2]{B_{#1}(#2)}
\newcommand{\llbracket}{[}
\newcommand{\rrbracket}{]}
\newcommand{\less}[1]{<_{#1}}
\newcommand{\greater}[1]{>_{#1}}
\newcommand{\leeq}[1]{{\leq}_{#1}}
\newcommand{\supr}[1]{\mathrm{sup}_{#1}}
\newcommand{\RR}{\mathbb{R}}
\newcommand{\QQ}{\mathbb{Q}}
\newcommand{\ZZ}{\mathbb{Z}}
\newcommand{\NN}{\mathbb{N}}

\begin{document}

\title{An SAD3 Formalization of Real Sequences for Walter Rudin's
\it{Principles of Mathematical Analysis}}

\author{Peter Koepke}

\date{August 25, 2018}

\maketitle


\section{Introduction}


\section{The Formalization}
\subsection{Some Set-Theoretic Terminology}


\begin{forthel}
	
	Let $NAT$ denote the set of natural numbers.
	Let $n$, $m$, $N$, $N1$, $N2$, $N3$ denote natural numbers.
	
	\begin{definition} Seq.	A sequence f is a function such that every element of Dom(f) is a natural number and every
	natural number is an element of $Dom(f)$ and for every $n \ f[n]$ is a real number.
	\end{definition}
	
	\begin{axiom} SequenceEq. Let $a, b$ be sequences. \\ $a = b$ iff for every $n \ a[n] = b[n]$.
	\end{axiom}
	
	\begin{definition} Convergence. Let $a$ be a sequence. Let $x$ be a real number. $a$ converges to $x$ iff for every positive real
	number $eps$ there exists $N$ such that for every $n$ such that $N < n \ dist(a[n],x) < eps$.
	\end{definition}
	
	\begin{definition} Conv. Let $a$ be a sequence. $a$ converges iff there exists a real number $x$ such that $a$ converges to $x$.
	\end{definition}
	
	\noindent Let $a$ is convergent stand for $a$ converges.
	\\Let $a$ diverges stand for not ($a$ converges).
	\\Let $a$ is divergent stand for $a$ diverges.
	
	\begin{definition} Range. Let $a$ be a sequence. $ran(a) = \{a[n] \mid \text{n is a natural number}\}$. 
	\end{definition}

	\begin{definition} RangeN. Let $a$ be a sequence. $ranN(a,N) = \{a[n] \mid \text{n is a natural number such that } n \leq N\}$.
	\end{definition}
	
	\begin{definition} FiniteRange.	Let $a$ be a sequence. $a$ has finite range iff there exists an $N$ such that $ran(a) = ranN(a,N)$.
	\end{definition}

	\begin{definition} InfiniteRange. Let $a$ be a sequence. $a$ has infinite range iff not ($a$ has finite range).
	\end{definition}
	
	\begin{lemma} Let $a$ be a sequence such that for every $n$
	((If $n = 0$ then $a[n] = 2$) and (If $n \neq 0$ then $a[n] = inv(n)$)).
	Then $a$ converges to $0$ and $a$ has infinite range.
	\end{lemma}
	\begin{proof} Let us show that $a$ converges to $0$.
	\begin{subproof}
	Let $eps$ be a positive real number. 
	Take $N$ such that $1 < N \cdot eps$ (by ArchimedeanAxiom, OnePos).
	
	Let us show that for every $n$ such that $N < n \ dist(a[n],0) < eps$.
	\begin{subproof}
	Assume $N < n$. Then $n \neq 0$.

	Let us show that $inv(n) < eps$.
	\begin{subproof}
	We have $N \cdot eps < n \cdot eps$ (by ComMult, MultInvariance).
	Hence $1 < n \cdot eps$ (by TransitivityOfOrder).
	$inv(n)$ is positive.
	Hence $inv(n) \cdot 1 < inv(n) \cdot (n \cdot eps)$ (by MultInvariance).
	We have $inv(n) \cdot 1 = inv(n)$ (by One).
	$inv(n) \cdot (n \cdot eps) .= (inv(n) \cdot n) \cdot eps$ (by AssMult)
	$.= 1 \cdot eps$ (by InvDummy)
	$.= eps$ (by OneDummy).
	\end{subproof}
	Hence $dist(a[n],0) = inv(n) < eps$.
	\end{subproof}
	\end{subproof}
	
	Let us show that $a$ has infinite range.
	Assume the contrary.
	Take $N$ such that $ran(a) = ranN(a,N)$ (by FiniteRange).
	Then $a[N + 1]$ is an element of $ran(a)$ (by OneNat, AddClosedNat, Range).
	
	Let us show that $a[N + 1]$ is not an element of $ranN(a,N)$.
	Let us show that for every $n$ such that $n \leq N \ a[n] \neq a[N + 1]$.
	Assume the contrary.
	Take $n$ such that $n \leq N \text{ and } a[n] = a[N + 1]$.
	Case $n = 0$.
	We have $2 = inv(N + 1)$.
	
	$(2 \cdot N) + 2 .= (2 \cdot N) + (2 \cdot 1)$ (by One)
	$.= 2 \cdot (N + 1)$ (by Distrib)
	$.= inv(N + 1) \cdot (N + 1)$
	$.= 1$ (by InvDummy).
	
	We have $(2 \cdot N) + 2 > 1$.
	Contradiction.
	end.
	Case $n \neq 0$.
	We have $inv(n) = inv(N + 1)$.
	Then $inv(inv(n)) = inv(inv(N + 1))$.
	
	We have $N + 1 \neq 0$.
	Hence $n = N + 1$ (by InvRule1).
	Contradiction.
	end.
	end.
	Hence $a[N + 1]$ is not an element of $ranN(a,N)$ (by RangeN).
	end.
	
	Contradiction.
	end.
	\end{proof}
	
	\begin{signature} Parity.
	$n$ is even is an atom.
	\end{signature}
	
	\noindent Let $n$ is odd stand for not ($n$ is even).
	
	\begin{axiom} ParityPlusOne.
	$n$ is even iff $n + 1$ is odd.
	\end{axiom}

	\begin{axiom} ZeroEven.
	$0$ is even.
	\end{axiom}
	
	\begin{lemma} OneOdd.
	$1$ is odd.
	\end{lemma} 
	
	\begin{lemma}
	Let $a$ be a sequence such that for every $n \ (((n \text{ is even}) \text{ then } a[n] = 1) \text{ and } ((\text{n is odd}) \text{ then } a[n] = -1))$.
	Then $a$ diverges and $a$ has finite range.
	\end{lemma}
	\begin{proof}
	Let us show that $a$ diverges.
	\begin{subproof}
	Assume the contrary.
	Take a real number $x$ such that $a$ converges to $x$.
	Take $N$ such that for every $n$ such that $N < n \ dist(a[n],x) < 1$ (by Convergence, OnePos).
	
	Let us show that $dist(a[N + 1],a[N + 2]) = 2$.
	\begin{subproof}
	Case $N + 1$ is even.
	Then $N + 2$ is odd.
	Hence $dist(a[N + 1],a[N + 2]) = dist(1,-1) = 2$.
	end.
	Case $N + 1$ is odd.
	Then $N + 2$ is even.
	Hence $dist(a[N + 1],a[N + 2]) = dist(-1,1) = 2$.
	end.
	\end{subproof}
	
	$a[N + 1]$ is a real number and $a[N + 2]$ is a real number.
	We have $dist(a[N + 1],a[N + 2]) \leq dist(a[N + 1],x) + dist(x,a[N + 2])$ (by DistTriangleIneq).
	We have $dist(x,a[N + 2]) = dist(a[N + 2],x)$ (by DistSymm).
	Hence $dist(a[N + 1],a[N + 2]) \leq dist(a[N + 1],x) + dist(a[N + 2],x)$.
	
	We have $dist(a[N + 1],x) < 1$ and $dist(a[N + 2],x) < 1$.
	Hence $dist(a[N + 1],x) + dist(a[N + 2],x) < 1 + 1$ (by AddInvariance).
	Hence $dist(a[N + 1],a[N + 2]) < 2$ (by MixedTransitivity).
	Hence $2 < 2$.
	Contradiction.
	\end{subproof}
	
	Let us show that $a$ has finite range.
	Let us show that $ran(a) = ranN(a,1)$.
	Let us show that every element of $ranN(a,1)$ is an element of $ran(a)$.
	Assume $x$ is an element of $ranN(a,1)$.
	Take $n$ such that $n \leq 1$ and $a[n] = x$ (by OneNat, RangeN).
	Hence $x$ is an element of $ran(a)$ (by Range).
	end.
	
	Let us show that every element of $ran(a)$ is an element of $ranN(a,1)$.
	Assume $x$ is an element of $ran(a)$.
	
	Let us show that $x = 1$ or $x = -1$.
	Take $n$ such that $a[n] = x$ (by Range).
	$n$ is even or $n$ is odd.
	Hence $x = 1$ or $x = -1$.
	end.
	
	We have $a[0] = 1$.
	We have $a[1] = -1$.
	
	Case $x = 1$.
	Then $x = a[0]$.
	We have $0 \leq 1$.
	Hence $x$ is an element of $ranN(a,1)$ (by ZeroNat, OneNat, RangeN).
	end.
	Case $x = -1$.
	Then $x = a[1]$.
	We have $1 \leq 1$.
	Hence $x$ is an element of $ranN(a,1)$ (by OneNat, RangeN).
	end.
	end.
	end.
	end.
	\end{proof}
	
\end{forthel}

\section{Remarks}
\subsection{Structures}


\subsection{Text Comparisons}



\section{{From \LaTeX} to ForTheL}



\section{Discussion}


\begin{thebibliography}{1}

\bibitem{Rudin}
  Walter Rudin,
  \textit{Principles of mathematical analysis},
  McGraw-Hill,
  1976.

\end{thebibliography}
  


\end{document}
