\documentclass{article}
\usepackage[english]{babel}
\usepackage{enumerate, latexsym, amssymb, amsmath}
\usepackage{framed, multicol}
\newenvironment{forthel}{\begin{leftbar}}{\end{leftbar}}

%%%%%%%%%% Start TeXmacs macros
\newcommand{\tmaffiliation}[1]{\\ #1}
\newcommand{\tmem}[1]{{\em #1\/}}
\newenvironment{enumeratenumeric}{\begin{enumerate}[1.] }{\end{enumerate}}
\newenvironment{proof}{\noindent\textbf{Proof\ }}{\hspace*{\fill}$\Box$\medskip}
\newenvironment{quoteenv}{\begin{quote} }{\end{quote}}
\newenvironment{subproof}{\begin{list}{}{}
		\item[\text{Proof}]}{\hfill $\surd$ \end{list}}
\newtheorem{axiom}{Axiom}
\newtheorem{lemma}{Lemma}
\newtheorem{theorem}{Theorem}
\newtheorem{definition}{Definition}
\newtheorem{signature}{Signature}
\newtheorem{proposition}{Proposition}
%%%%%%%%%% End TeXmacs macros

\newcommand{\event}{UITP 2018}
\newcommand{\dom}{Dom}
\newcommand{\fun}{aFunction}
\newcommand{\sym}{sym}
\newcommand{\halfline}{{\vspace{3pt}}}
\newcommand{\tab}{{\hspace{1cm}}}
\newcommand{\ball}[2]{B_{#1}(#2)}
\newcommand{\llbracket}{[}
\newcommand{\rrbracket}{]}
\newcommand{\less}[1]{<_{#1}}
\newcommand{\greater}[1]{>_{#1}}
\newcommand{\leeq}[1]{{\leq}_{#1}}
\newcommand{\supr}[1]{\mathrm{sup}_{#1}}
\newcommand{\RR}{\mathbb{R}}
\newcommand{\QQ}{\mathbb{Q}}
\newcommand{\ZZ}{\mathbb{Z}}
\newcommand{\NN}{\mathbb{N}}

\begin{document}

\title{An SAD3 Formalization of Real Sequences for Walter Rudin's
\it{Principles of Mathematical Analysis}}

\author{Lukas Schembecker, Annika Hennes}

\date{September 10, 2018}

\maketitle


\section{Introduction}
The present formalization deals with chapter 3 'Numerical Sequences and Series' (pages 47 to 58). We only focus on the part about Sequences and regard the special case of the metric space $\mathbb{R}$.


\section{The Formalization}
\subsection{Some Set-Theoretic Terminology}


\begin{forthel}
	
	Let $n$, $m$, $N$, $N1$, $N2$, $N3$ denote natural numbers.
	
	\begin{definition}
		$NAT$ is the set of natural numbers.
	\end{definition}
	
	\begin{definition}[Seq]	A sequence f is a function such that every element of Dom(f) is a natural number and every
	natural number is an element of $Dom(f)$ and for every $n \ f[n]$ is a real number.
	\end{definition}
	
	\begin{axiom} [SequenceEq] Let $a, b$ be sequences. \\ $a = b$ iff for every $n \ a[n] = b[n]$.
	\end{axiom}
	
	\begin{definition} [Convergence] Let $a$ be a sequence. Let $x$ be a real number. $a$ converges to $x$ iff for every positive real
	number $eps$ there exists $N$ such that for every $n$ such that $N < n \ dist(a[n],x) < eps$.
	\end{definition}
	
	\begin{definition} [Conv] Let $a$ be a sequence. $a$ converges iff there exists a real number $x$ such that $a$ converges to $x$.
	\end{definition}
	
	\noindent Let $a$ is convergent stand for $a$ converges.
	\\Let $a$ diverges stand for not ($a$ converges).
	\\Let $a$ is divergent stand for $a$ diverges.
	
	\begin{definition} [Range] Let $a$ be a sequence. $ran(a) = \{a[n] \mid \text{n is a natural number}\}$. 
	\end{definition}

	\begin{definition} [RangeN] Let $a$ be a sequence. $ranN(a,N) = \{a[n] \mid \text{n is a natural number such that } n \leq N\}$.
	\end{definition}
	
	\begin{definition} [FiniteRange]	Let $a$ be a sequence. $a$ has finite range iff there exists an $N$ such that $ran(a) = ranN(a,N)$.
	\end{definition}

	\begin{definition} [InfiniteRange] Let $a$ be a sequence. $a$ has infinite range iff not ($a$ has finite range).
	\end{definition}
	
	\begin{lemma} Let $a$ be a sequence such that for every $n$
	((If $n = 0$ then $a[n] = 2$) and (If $n \neq 0$ then $a[n] = inv(n)$)).
	Then $a$ converges to $0$ and $a$ has infinite range.
	\end{lemma}
	\begin{proof} Let us show that $a$ converges to $0$.
	\begin{subproof}
	Let $eps$ be a positive real number. 
	Take $N$ such that $1 < N \cdot eps$ (by ArchimedeanAxiom, OnePos).
	
	Let us show that for every $n$ such that $N < n \ dist(a[n],0) < eps$.
	\begin{subproof}
	Assume $N < n$. Then $n \neq 0$.

	Let us show that $inv(n) < eps$.
	\begin{subproof}
	We have $N \cdot eps < n \cdot eps$ (by ComMult, MultInvariance).
	Hence $1 < n \cdot eps$ (by TransitivityOfOrder).
	$inv(n)$ is positive.
	Hence $inv(n) \cdot 1 < inv(n) \cdot (n \cdot eps)$ (by MultInvariance).
	We have $inv(n) \cdot 1 = inv(n)$ (by One).
	$inv(n) \cdot (n \cdot eps) .= (inv(n) \cdot n) \cdot eps$ (by AssMult)
	$.= 1 \cdot eps$ (by InvDummy)
	$.= eps$ (by OneDummy).
	\end{subproof}
	Hence $dist(a[n],0) = inv(n) < eps$.
	\end{subproof}
	\end{subproof}
	
	Let us show that $a$ has infinite range.
	\begin{subproof}
	Assume the contrary.
	Take $N$ such that $ran(a) = ranN(a,N)$ (by FiniteRange).
	Then $a[N + 1]$ is an element of $ran(a)$ (by OneNat, AddClosedNat, Range).
	
	Let us show that $a[N + 1]$ is not an element of $ranN(a,N)$.
	\begin{subproof}
	Let us show that for every $n$ such that $n \leq N \ a[n] \neq a[N + 1]$.
	\begin{subproof}
	Assume the contrary.
	Take $n$ such that $n \leq N \text{ and } a[n] = a[N + 1]$.
	Case $n = 0$.
	We have $2 = inv(N + 1)$.
	
	$(2 \cdot N) + 2 .= (2 \cdot N) + (2 \cdot 1)$ (by One)
	$.= 2 \cdot (N + 1)$ (by Distrib)
	$.= inv(N + 1) \cdot (N + 1)$
	$.= 1$ (by InvDummy).
	
	We have $(2 \cdot N) + 2 > 1$.
	Contradiction.
	end.
	Case $n \neq 0$.
	We have $inv(n) = inv(N + 1)$.
	Then $inv(inv(n)) = inv(inv(N + 1))$.
	
	We have $N + 1 \neq 0$.
	Hence $n = N + 1$ (by InvRule1).
	Contradiction.
	end.
	\end{subproof}
	Hence $a[N + 1]$ is not an element of $ranN(a,N)$ (by RangeN).
	\end{subproof}
	
	Contradiction.
	\end{subproof}
	\end{proof}
	
	\begin{signature} [Parity]
	$n$ is even is an atom.
	\end{signature}
	
	\noindent Let $n$ is odd stand for not ($n$ is even).
	
	\begin{axiom} [ParityPlusOne]
	$n$ is even iff $n + 1$ is odd.
	\end{axiom}

	\begin{axiom} [ZeroEven]
	$0$ is even.
	\end{axiom}
	
	\begin{lemma} [OneOdd]
	$1$ is odd.
	\end{lemma} 
	
	\begin{lemma}
	Let $a$ be a sequence such that for every $n \ (((n \text{ is even}) \text{ then } a[n] = 1) \text{ and } ((\text{n is odd}) \text{ then } a[n] = -1))$.
	Then $a$ diverges and $a$ has finite range.
	\end{lemma}
	\begin{proof}
	Let us show that $a$ diverges.
	\begin{subproof}
	Assume the contrary.
	Take a real number $x$ such that $a$ converges to $x$.
	Take $N$ such that for every $n$ such that $N < n \ dist(a[n],x) < 1$ (by Convergence, OnePos).
	
	Let us show that $dist(a[N + 1],a[N + 2]) = 2$.
	\begin{subproof}
	Case $N + 1$ is even.
	Then $N + 2$ is odd.
	Hence $dist(a[N + 1],a[N + 2]) = dist(1,-1) = 2$.
	end.
	Case $N + 1$ is odd.
	Then $N + 2$ is even.
	Hence $dist(a[N + 1],a[N + 2]) = dist(-1,1) = 2$.
	end.
	\end{subproof}
	
	$a[N + 1]$ is a real number and $a[N + 2]$ is a real number.
	We have $dist(a[N + 1],a[N + 2]) \leq dist(a[N + 1],x) + dist(x,a[N + 2])$ (by DistTriangleIneq).
	We have $dist(x,a[N + 2]) = dist(a[N + 2],x)$ (by DistSymm).
	Hence $dist(a[N + 1],a[N + 2]) \leq dist(a[N + 1],x) + dist(a[N + 2],x)$.
	
	We have $dist(a[N + 1],x) < 1$ and $dist(a[N + 2],x) < 1$.
	Hence $dist(a[N + 1],x) + dist(a[N + 2],x) < 1 + 1$ (by AddInvariance).
	Hence $dist(a[N + 1],a[N + 2]) < 2$ (by MixedTransitivity).
	Hence $2 < 2$.
	Contradiction.
	\end{subproof}
	
	Let us show that $a$ has finite range.
	\begin{subproof}
	Let us show that $ran(a) = ranN(a,1)$.
	\begin{subproof}
	Let us show that every element of $ranN(a,1)$ is an element of $ran(a)$.
	\begin{subproof}
	Assume $x$ is an element of $ranN(a,1)$.
	Take $n$ such that $n \leq 1$ and $a[n] = x$ (by OneNat, RangeN).
	Hence $x$ is an element of $ran(a)$ (by Range).
	\end{subproof}
	
	Let us show that every element of $ran(a)$ is an element of $ranN(a,1)$.
	\begin{subproof}
	Assume $x$ is an element of $ran(a)$.
	
	Let us show that $x = 1$ or $x = -1$.
	\begin{subproof}
	Take $n$ such that $a[n] = x$ (by Range).
	$n$ is even or $n$ is odd.
	Hence $x = 1$ or $x = -1$.
	\end{subproof}
	
	We have $a[0] = 1$.
	We have $a[1] = -1$.
	
	Case $x = 1$.
	Then $x = a[0]$.
	We have $0 \leq 1$.
	Hence $x$ is an element of $ranN(a,1)$ (by ZeroNat, OneNat, RangeN).
	end.
	Case $x = -1$.
	Then $x = a[1]$.
	We have $1 \leq 1$.
	Hence $x$ is an element of $ranN(a,1)$ (by OneNat, RangeN).
	end.
	\end{subproof}
	\end{subproof}
	\end{subproof}
	\end{proof}
	
	\begin{definition} [Neighb]
	Let $eps$ be a positive real number. Let $x$ be a real number.
	$Neighb(x,eps) = \{y | y \text{ is a real number such that } dist(y,x) < eps\}$.
	\end{definition}
	
	\begin{theorem} [ConvNeighborhood]
	Let $a$ be a sequence. Let $x$ be a real number.
	$a$ converges to $x$ iff for every positive real number $eps$ there exists a $N$
	such that for every $n$ such that $N < n \ a[n]$ is an element of $Neighb(x,eps)$.
	\end{theorem}
	\begin{proof}
	Let us show that (if $a$ converges to $x$ then for every positive real number $eps$ there exists a $N$
	such that for every $n$ such that $N < n \ a[n]$ is an element of $Neighb(x,eps)$).
	\begin{subproof}
	Assume $a$ converges to $x$.
	Let $eps$ be a positive real number.
	Take $N$ such that for every $n$ such that $N < n \ dist(a[n],x) < eps$ (by Convergence).
	Hence for every $n$ such that $N < n \ a[n]$ is an element of $Neighb(x,eps)$ (by Neighb).
	\end{subproof}
	
	Let us show that (if for every positive real number $eps$ there exists a $N$ such that
	for every $n$ such that $N < n \ a[n]$ is an element of $Neighb(x,eps)$ then $a$ converges to $x$).
	\begin{subproof}
	Assume for every positive real number $eps$ there exists a $N$ such that
	for every $n$ such that $N < n \ a[n]$ is an element of $Neighb(x,eps)$.
	Let $eps$ be a positive real number.
	Take $N$ such that for every $n$ such that $N < n \ a[n]$ is an element of $Neighb(x,eps)$.
	Hence for every $n$ such that $N < n \ dist(a[n],x) < eps$ (by Neighb).
	\end{subproof}
	\end{proof}	
	
	\begin{lemma} [DistEqual]
	Let $x$ and $y$ be real numbers. Assume for every positive real number $eps$ $dist(x,y) < eps$.
	Then $x = y$.
	\end{lemma}
	\begin{proof}
	Assume the contrary.
	Then there exists a positive real number $eps2$ such that $eps2 < dist(x,y)$.
	A contradiction.
	\end{proof}
	
	\begin{theorem} [LimitUnique]
	Let $a$ be a sequence. Let $x, y$ be real numbers. Assume $a$ converges to $x$ and $a$ converges to $y$.
	Then $x = y$.
	\end{theorem}
	\begin{proof}
	Let us show that for every positive real number $eps \ dist(x,y) < eps$.
	\begin{subproof}
	Let $eps$ be a positive real number.
	Take a positive real number $halfeps$ such that $halfeps = inv(2) \cdot eps$.
	
	Take $N1$ such that for every $n$ such that $N1 < n \ dist(a[n],x) < halfeps$ (by Convergence).
	Take $N2$ such that for every $n$ such that $N2 < n \ dist(a[n],y) < halfeps$ (by Convergence).
	
	For every $n \ dist(x,y) \leq dist(x,a[n]) + dist(a[n],y)$ (by DistTriangleIneq). 
	Take $N3$ such that $N3 = max(N1,N2) + 1$.
	Then $N1 < N3$ and $N2 < N3$.
	
	Hence $dist(x,a[N3]) < halfeps$ and $dist(a[N3],y) < halfeps$ (by DistSymm).
	Hence $dist(x,a[N3]) + dist(a[N3],y) < halfeps + halfeps$ (by AddInvariance).
	Hence $dist(x,y) < halfeps + halfeps$ (by MixedTransitivity).
	Hence $dist(x,y) < eps$ (by TwoHalf).
	\end{subproof}
	Therefore $x = y$ (by DistEqual).
	\end{proof}
	
	\begin{definition} [BoundedBy]
	Let $a$ be a sequence. Let $K$ be a real number. $a$ is bounded by $K$ iff
	for every $n \ abs(a[n]) \leq K$.
	\end{definition}
	
	\begin{definition} [BoundedSequence]
	Let $a$ be a sequence. $a$ is bounded iff there exists a real number $K$ such that
	$a$ is bounded by $K$.
	\end{definition}
	
	\begin{signature} [MaxN]
	Let $a$ be a sequence. $maxN(a,N)$ is a real number such that
	(there exists $n$ such that $n \leq N$ and $maxN(a,N) = a[n]$) and
	(for every $n$ such that $n \leq N \ a[n] =< maxN(a,N)$).
	\end{signature}
	
	\begin{theorem} [ConvergentImpBounded]
	Let $a$ be a sequence. Assume that $a$ converges. Then $a$ is bounded.
	\end{theorem}
	\begin{proof}
	Take a real number $x$ such that $a$ converges to $x$.
	Take $N$ such that for every $n$ such that $N < n \ dist(a[n],x) < 1$ (by Convergence, OnePos).
	Define $b[k] = abs(a[k])$ for $k$ in $NAT$.
	Take a real number $K$ such that $K = max(1 + abs(x), maxN(b,N))$.
	Let us show that $a$ is bounded by $K$.
	\begin{subproof}
	Let us show that for every $n \ abs(a[n]) \leq K$.
	\begin{subproof} 
	Let $n$ be a natural number.
	We have $n \leq N$ or $n > N$.
	Case $n =< N$.
	We have $abs(a[n]) = b[n] \leq maxN(b,N)$ (by MaxN).
	We have $maxN(b,N) \leq K$ (by MaxIneqDummy).
	Therefore $abs(a[n]) \leq K$ (by LeqTransitivity).
	end.
	Case $n > N$.
	We have $dist(a[n],x) < 1$.
	We have $1 + abs(x) \leq K$ (by MaxIneq).
	
	$abs(a[n]) .= abs(a[n] + 0)$ (by Zero)
	$.= abs(a[n] + (x - x))$ (by Neg)
	$.= abs(a[n] + ((-x) + x))$ (by ComAdd)
	$.= abs((a[n] - x) + x)$ (by AssAdd).
	
	Hence $abs(a[n]) \leq abs(a[n] - x) + abs(x)$ (by AbsTriangleIneq).
	Hence $abs(a[n]) \leq dist(a[n],x) + abs(x)$.
	
	We have $dist(a[n],x) + abs(x) < 1 + abs(x)$ (by MixedAddInvariance).
	Hence $abs(a[n]) \leq 1 + abs(x)$ (by MixedTransitivity).
	Therefore $abs(a[n]) \leq K$ (by LeqTransitivity).
	end.
	\end{subproof}
	Hence $a$ is bounded by $K$ (by BoundedBy).
	\end{subproof}
	\end{proof}
	
	\begin{definition} [LimitPointOfSet]
	Let $E$ be a set. Assume every element of $E$ is a real number. A limit point of $E$
	is a real number $x$ such that for every positive real number $eps$ there exists an element
	$y$ of $E$ such that $y$ is an element of $Neighb(x,eps)$ and $y \neq x$.
	\end{definition}
	
	\begin{theorem} [ConvLimitPoint]
	Let $E$ be a set. Assume every element of $E$ is a real number. Let $x$ be a limit point of $E$.
	Then there exists a sequence $a$ such that $a$ converges to $x$ and for every $n \ a[n]$ is an element of $E$.
	\end{theorem}
	\begin{proof}
	Let us show that for every $n$ such that $n > 0$ there exists an element $y$ of $E$ such that
	$y$ is an element of $Neighb(x,inv(n))$ and $y \neq x$.
	\begin{subproof}
	Assume $n > 0$.
	Then $inv(n)$ is a positive real number.
	Take an element $y$ of $E$ such that $y$ is an element of $Neighb(x,inv(n))$
	and $y \neq x$ (by LimitPointOfSet).
	\end{subproof}
	Define $a[n] =$ Case $n = 0$ $->$ Choose an element $y$ of $E$ such that $y$ is an element of
	$Neighb(x,1)$ and $y \neq x$ in $y$,
	Case $n > 0$ $->$ Choose an element $y$ of $E$ such that $y$ is an element of
	$Neighb(x,inv(n))$ and $y \neq x$ in $y$
	for $n$ in $NAT$.
	$a$ is a sequence.	
	Then for every $n \ a[n]$ is an element of $E$.
	Let us show that $a$ converges to $x$.
	\begin{subproof}
	Let $eps$ be a positive real number.
	Take $N$ such that $1 < N \cdot eps$ (by ArchimedeanAxiom, OnePos).

	Let us show that for every $n$ such that $N < n \ dist(a[n],x) < eps$.
	\begin{subproof}
	Assume $N < n$. Then $n \neq 0$.
	Then $a[n]$ is an element of $E$ such that $a[n]$ is an element of $Neighb(x,inv(n))$.
	Hence $dist(a[n],x) < inv(n)$.
	
	Let us show that $inv(n) < eps$.
	\begin{subproof}
	We have $N \cdot eps < n \cdot eps$ (by ComMult, MultInvariance).
	Hence $1 < n \cdot eps$ (by TransitivityOfOrder).
	$inv(n)$ is positive.
	Hence $inv(n) \cdot 1 < inv(n) \cdot (n \cdot eps)$ (by MultInvariance).
	We have $inv(n) \cdot 1 = inv(n)$ (by One).
	$inv(n) \cdot (n \cdot eps) .= (inv(n) \cdot n) \cdot eps$ (by AssMult)
	$.= 1 \cdot eps$ (by InvDummy)
	$.= eps$ (by OneDummy).
	\end{subproof}
	Hence $dist(a[n],x) < eps$ (by TransitivityOfOrder).
	\end{subproof}
	\end{subproof}
	\end{proof}
	
	\begin{definition} [SequenceSum]
	Let $a,b$ be sequences. $a +' b$ is a sequence such that for every $n \ (a +' b)[n] = a[n] + b[n]$.
	\end{definition}
	
	\begin{definition}[SequenceProd]
	Let $a,b$ be sequences. $a *' b$ is a sequence such that for every $n \ (a *' b)[n] = a[n] \cdot b[n]$.
	\end{definition}
	
	\begin{definition} [SequenceConstSum]
	Let $a$ be a sequence. Let $c$ be a real number. $c +'' a$ is a sequence such that for every $n \ (c +'' a)[n] = c + a[n]$.
	\end{definition}
	
	\begin{definition} [SequenceConstProd]
	Let $a$ be a sequence. Let $c$ be a real number. $c *'' a$ is a sequence such that for every $n \ (c *'' a)[n] = c \cdot a[n]$.
	\end{definition}

	\begin{definition} [SequenceDiv]
	Let $a$ be a sequence. Assume for every $n \ a[n] \neq 0$. $div(a)$ is a sequence such that for every $n \ (div(a))[n] = inv(a[n])$.
	\end{definition}
	
	\begin{theorem} [SumConv]
	Let $a,b$ be sequences. Let $x,y$ be real numbers. Assume $a$ converges to $x$ and $b$ converges to $y$.
	Then $a +' b$ converges to $x + y$.
	\end{theorem}
	\begin{proof}.
	Let $eps$ be a positive real number.
	Take a positive real number $halfeps$ such that $halfeps = inv(2) \cdot eps$.
	Take $N1$ such that for every $n$ such that $N1 < n \ dist(a[n],x) < halfeps$ (by Convergence).
	Take $N2$ such that for every $n$ such that $N2 < n \ dist(b[n],y) < halfeps$ (by Convergence).
	Take $N$ such that $N = max(N1,N2)$.
	Then $N1 \leq N$ and $N2 \leq N$.
	Let us show that for every $n$ such that $N < n \ dist((a +' b)[n],(x+y)) < eps$.
	\begin{subproof}
	Assume $N < n$.
	We have $dist(a[n],x) < halfeps$.
	We have $dist(b[n],y) < halfeps$.
	$abs((a[n] + b[n]) - (x + y)) .= abs((a[n] + b[n]) + ((-x) + (-y)))$ (by MinusRule1)
	$.= abs((-x) + ((a[n] + b[n]) - y))$ (by BubbleAdd)
	$.= abs((-x) + (a[n] + (b[n] - y)))$ (by AssAdd)
	$.= abs(((-x) + a[n]) + (b[n] - y))$ (by AssAdd)
	$.= abs((a[n] - x) + (b[n] - y))$ (by ComAdd).
	We have $abs((a[n] - x) + (b[n] - y)) \leq abs(a[n] - x) + abs(b[n] - y)$  (by AbsTriangleIneq).
	Hence $abs((a[n] + b[n]) - (x + y)) \leq dist(a[n],x) + dist(b[n],y)$.
	Hence $dist(a[n],x) + dist(b[n],y) < halfeps + halfeps$ (by AddInvariance).
	Hence $abs((a[n] + b[n]) - (x + y)) < halfeps + halfeps$ (by MixedTransitivity).
	Hence $dist((a +' b)[n],(x + y)) < halfeps + halfeps$.
	Hence $dist((a +' b)[n],(x + y)) < eps$ (by TwoHalf).
	\end{subproof}
	\end{proof}
	
	\begin{lemma} [ConstConv]
	Let $c$ be a real number. Let $cn$ be a sequence such that for every $n \ cn[n] = c$.
	Then $cn$ converges to $c$.
	\end{lemma}
	\begin{proof}
	Let $eps$ be a positive real number.
	Let us show that for every $n$ such that $0 < n \ dist(cn[n],c) < eps$.
	\begin{subproof}
	Assume $0 < n$.
	$dist(cn[n],c) = dist(c,c) = 0$ (by IdOfInd).
	Hence $dist(cn[n],c) < eps$.
	\end{subproof}
	\end{proof}
	
	\begin{theorem} [SumConstConv]
	Let $a$ be a sequence. Let $x,c$ be real numbers. Assume $a$ converges to $x$.
	Then $c +'' a$ converges to $c + x$.
	\end{theorem}
	\begin{proof}
	Define $cn[n] = c$ for $n$ in $NAT$.
	$cn$ is a sequence.
	Let us show that $c +'' a = (cn +' a)$.
	\begin{subproof}
	Let us show that for every $n \ (c +'' a)[n] = (cn +' a)[n]$.
	\begin{subproof}
	Let $n$ be a natural number.
	$(c +'' a)[n] .= c + a[n]
	.= cn[n] + a[n]
	.= (cn +' a)[n]$.
	\end{subproof}
	Hence $c +'' a = (cn +' a)$ (by SequenceEq).
	\end{subproof}
	$cn$ converges to $c$ (by ConstConv).
	Then $c +'' a$ converges to $c + x$ (by SumConv).
	\end{proof}
	
	\begin{theorem} [ProdConstConv]
	Let $a$ be a sequence. Let $x,c$ be real numbers. Assume $a$ converges to $x$.
	Then $c *'' a$ converges to $c \cdot x$.
	\end{theorem}
	\begin{proof}
	Case $c = 0$.
	We have $c \cdot x = 0$.
	Let us show that for every $n \ (c *'' a)[n] = 0$. 
	\begin{subproof}
	$(c *'' a)[n] .= c * a[n]
	.= 0 * a[n]
	.= 0$ (by ZeroMult).
	\end{subproof}
	Hence $c *'' a$ converges to $c \cdot x$ (by ConstConv).
	end.
	Case $c \neq 0$.
	Let $eps$ be a positive real number. 
	(1)     Take a positive real number $invEps$ such that $invEps = inv(abs(c)) \cdot eps$.
	Take $N$ such that for every $n$ such that $N < n \ dist(a[n],x) < invEps$ (By Convergence).
	
	Let us show that for every $n$ such that $N < n \ dist(c \cdot a[n],c \cdot x) < eps$.
	\begin{subproof}
	Assume $N < n$.
	$abs((c \cdot a[n]) - (c \cdot x)) .= abs((c \cdot a[n]) +  ((-1) \cdot (c \cdot x))) (by MinusRule4)$ 
	$.= abs((c \cdot a[n]) + (((-1) \cdot c) \cdot x))$ (by AssMult)
	$.= abs((c \cdot a[n]) + ((c \cdot (-1)) \cdot x))$ (by ComMult)
	$.= abs((c \cdot a[n]) + (c \cdot ((-1) \cdot x)))$ (by AssMult)
	$.= abs((c \cdot a[n]) + (c \cdot (-x)))$ (by MinusRule4)
	$.= abs(c \cdot (a[n] - x))$ (by Distrib)
	$.= abs(c) \cdot abs(a[n] - x)$ (by AbsMult).
	$abs(c) \cdot dist(a[n],x) < abs(c) \cdot invEps$ (by AbsPos, MultInvariance).
	Hence $abs((c \cdot a[n]) - (c \cdot x)) < abs(c) \cdot invEps$.
	
	$abs(c) \cdot invEps .= abs(c) \cdot (inv(abs(c)) \cdot eps)$ (by 1)
	$.= (abs(c) \cdot inv(abs(c))) \cdot eps$ (by AssMult)
	$.= 1 \cdot eps$ (by Inverse)
	$.= eps$ (by OneDummy).
	
	Hence $dist(c \cdot a[n],c \cdot x) < eps$.
	\end{subproof}
	end.
	\end{proof}

\end{forthel}

\section{Remarks}
\subsection{Structures}


\subsection{Text Comparisons}



\section{{From \LaTeX} to ForTheL}



\section{Discussion}


\begin{thebibliography}{1}

\bibitem{Rudin}
  Walter Rudin,
  \textit{Principles of mathematical analysis},
  McGraw-Hill,
  1976.

\end{thebibliography}
  


\end{document}
