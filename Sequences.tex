\documentclass{article}
\usepackage[english]{babel}
\usepackage{enumerate, latexsym, amssymb, amsmath}
\usepackage{framed, multicol}
\newenvironment{forthel}{\begin{leftbar}}{\end{leftbar}}

%%%%%%%%%% Start TeXmacs macros
\newcommand{\tmaffiliation}[1]{\\ #1}
\newcommand{\tmem}[1]{{\em #1\/}}
\newenvironment{enumeratenumeric}{\begin{enumerate}[1.] }{\end{enumerate}}
\newenvironment{proof}{\noindent\textbf{Proof\ }}{\hspace*{\fill}$\Box$\medskip}
\newenvironment{quoteenv}{\begin{quote} }{\end{quote}}
\newtheorem{axiom}{Axiom}
\newtheorem{lemma}{Lemma}
\newtheorem{theorem}{Theorem}
\newtheorem{definition}{Definition}
\newtheorem{signature}{Signature}
\newtheorem{proposition}{Proposition}
%%%%%%%%%% End TeXmacs macros

\newcommand{\event}{UITP 2018}
\newcommand{\dom}{Dom}
\newcommand{\fun}{aFunction}
\newcommand{\sym}{sym}
\newcommand{\halfline}{{\vspace{3pt}}}
\newcommand{\tab}{{\hspace{1cm}}}
\newcommand{\ball}[2]{B_{#1}(#2)}
\newcommand{\llbracket}{[}
\newcommand{\rrbracket}{]}
\newcommand{\less}[1]{<_{#1}}
\newcommand{\greater}[1]{>_{#1}}
\newcommand{\leeq}[1]{{\leq}_{#1}}
\newcommand{\supr}[1]{\mathrm{sup}_{#1}}
\newcommand{\RR}{\mathbb{R}}
\newcommand{\QQ}{\mathbb{Q}}
\newcommand{\ZZ}{\mathbb{Z}}
\newcommand{\NN}{\mathbb{N}}

\begin{document}

\title{An SAD3 Formalization of Real Sequences for Walter Rudin's
\it{Principles of Mathematical Analysis}}

\author{Peter Koepke}

\date{August 25, 2018}

\maketitle


\section{Introduction}


\section{The Formalization}
\subsection{Some Set-Theoretic Terminology}


\begin{forthel}
	
	Let $NAT$ denote the set of natural numbers.
	Let $n, m, N, N1, N2, N3$ denote natural numbers.
	
	Sequences
	
	\begin{definition} Seq.
	A sequence f is a function such that every element of Dom(f) is a natural number and every
	natural number is an element of $Dom(f)$ and for every $n f[n]$ is a real number.
	\end{definition}
	
	\begin{axiom} SequenceEq.
	Let $a, b$ be sequences. $a = b$ iff (for every $n a[n] = b[n]$).
	\end{axiom}
	
\end{forthel}

\section{Remarks}
\subsection{Structures}


\subsection{Text Comparisons}



\section{{From \LaTeX} to ForTheL}



\section{Discussion}


\begin{thebibliography}{1}

\bibitem{Rudin}
  Walter Rudin,
  \textit{Principles of mathematical analysis},
  McGraw-Hill,
  1976.

\end{thebibliography}
  


\end{document}
