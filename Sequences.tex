\documentclass{article}
\usepackage[english]{babel}
\usepackage{enumerate, latexsym, amssymb, amsmath}
\usepackage{framed, multicol}
\newenvironment{forthel}{\begin{leftbar}}{\end{leftbar}}

%%%%%%%%%% Start TeXmacs macros
\newcommand{\tmaffiliation}[1]{\\ #1}
\newcommand{\tmem}[1]{{\em #1\/}}
\newenvironment{enumeratenumeric}{\begin{enumerate}[1.] }{\end{enumerate}}
\newenvironment{proof}{\noindent\textbf{Proof\ }}{\hspace*{\fill}$\Box$\medskip}
\newenvironment{quoteenv}{\begin{quote} }{\end{quote}}
\newenvironment{subproof}{\begin{list}{}{}
		\item[\text{Proof}]}{\hfill $\surd$ \end{list}}
\newenvironment{case}{\begin{list}{}{}
		\item[]}{\end{list}}	
\newtheorem{axiom}{Axiom}
\newtheorem{lemma}{Lemma}\plustwo
\newtheorem{theorem}{Theorem}
\newtheorem{definition}{Definition}
\newtheorem{signature}{Signature}
\newtheorem{proposition}{Proposition}
%%%%%%%%%% End TeXmacs macros

\newcommand{\event}{UITP 2018}
\newcommand{\dom}{Dom}
\newcommand{\fun}{aFunction}
\newcommand{\sym}{sym}
\newcommand{\halfline}{{\vspace{3pt}}}
\newcommand{\tab}{{\hspace{1cm}}}
\newcommand{\ball}[2]{B_{#1}(#2)}
\newcommand{\llbracket}{[}
\newcommand{\rrbracket}{]}
\newcommand{\less}[1]{<_{#1}}
\newcommand{\greater}[1]{>_{#1}}
\newcommand{\leeq}[1]{{\leq}_{#1}}
\newcommand{\supr}[1]{\mathrm{sup}_{#1}}
\newcommand{\RR}{\mathbb{R}}
\newcommand{\QQ}{\mathbb{Q}}
\newcommand{\ZZ}{\mathbb{Z}}
\newcommand{\NN}{\mathbb{N}}
\newcommand{\cdotone}{\cdot}
\newcommand{\cdottwo}{\cdot}
\newcommand{\plusone}{+}
\newcommand{\plustwo}{+}
\newcommand{\halfeps}{\frac{\epsilon}{2}}
\newcommand{\inveps}{\frac{1}{\epsilon}}
\newcommand{\rooteps}{\sqrt{\epsilon}}


\begin{document}

\title{An SAD3 Formalization of Real Sequences for Walter Rudin's
\it{Principles of Mathematical Analysis}}

\author{Lukas Schembecker, Annika Hennes}

\date{September 10, 2018}

\maketitle


\section{Introduction}
The present formalization deals with chapter 3 'Numerical Sequences and Series' (pages 47 to 58) of the {\it Principles of Mathematical Analysis} by Walter Rudin \cite{Rudin}. We only focus on the part about Sequences and regard the special case of the metric space $\mathbb{R}$. Beside using some of the lemmata provided in 'examples/Forster/Reals.ftl' we also had to think of some own helpful propositions. In order to improve the readability, we moved most of the lemmata, which only serve as expedients for the theorems presented in Rudin, to an extra file - 'helper.ftl'. $\rooteps$.


\section{The Formalization}
\subsection{Sequences.ftl}


\begin{forthel}
	\noindent $\text{[read Sequences/Naturals.ftl]}\\
	\text{[read Sequences/helper.ftl]}\\
	\text{[sequence/-s]}\\
	\text{[converge/-s]}$\\
	Let $n$, $m$, $k$, $N$, $N1$, $N2$, $N3$ denote natural numbers.

	\begin{definition}
		$\NN$ is the set of natural numbers.
	\end{definition}
	
	\begin{definition}[Seq]	A sequence f is a function such that every element of Dom(f) is a natural number and every
	natural number is an element of $Dom(f)$ and for every $n \ f[n]$ is a real number.
	\end{definition}
	
	\begin{axiom}[SequenceEq] Let $a, b$ be sequences. \\ $a = b$ iff for every $n \ a[n] = b[n]$.
	\end{axiom}
	
	\begin{definition}[Convergence] Let $a$ be a sequence. Let $x$ be a real number. $a$ converges to $x$ iff for every positive real
	number $\epsilon$ there exists $N$ such that for every $n$ such that $N < n \ dist(a[n],x) < \epsilon$.
	\end{definition}
	
	\begin{definition}[Conv] Let $a$ be a sequence. $a$ converges iff there exists a real number $x$ such that $a$ converges to $x$.
	\end{definition}
	
	\noindent Let $a$ is convergent stand for $a$ converges.
	\\Let $a$ diverges stand for not ($a$ converges).
	\\Let $a$ is divergent stand for $a$ diverges.
	
	\begin{definition}[Range] Let $a$ be a sequence. \\ $ran(a) = \{a[n] \mid \text{n is a natural number} \}$. 
	\end{definition}

	\begin{definition}[RangeN] Let $a$ be a sequence. \\ $ranN(a,N) = \{a[n] \mid \text{n is a natural number such that } n \leq N\}$. 
	\end{definition}
	
	\begin{definition}[FiniteRange]	Let $a$ be a sequence. $a$ has finite range iff there exists an $N$ such that $ran(a) = ranN(a,N)$.
	\end{definition}

	\begin{definition}[InfiniteRange] Let $a$ be a sequence. $a$ has infinite range iff not ($a$ has finite range).
	\end{definition}
	
	\begin{lemma} Let $a$ be a sequence such that for every $n$
	((If $n = 0$ then $a[n] = 2$) and (If $n \neq 0$ then $a[n] = inv(n)$)).
	Then $a$ converges to $0$ and $a$ has infinite range.
	\end{lemma}
	\begin{proof} Let us show that $a$ converges to $0$.
	\begin{subproof}
	Let $\epsilon$ be a positive real number. 
	Take $N$ such that $1 < N \cdot \epsilon$ (by ArchimedeanAxiom, OnePos).
	Let us show that for every $n$ such that $N < n \ dist(a[n],0) < \epsilon$.
	\begin{subproof}
	Assume $N < n$. Then $n \neq 0$.
	Let us show that $inv(n) < \epsilon$.
	\begin{subproof}
	We have $N \cdot \epsilon < n \cdot \epsilon$ (by ComMult, MultInvariance).
	Hence $1 < n \cdot \epsilon$ (by TransitivityOfOrder).
	$inv(n)$ is positive.
	Hence $inv(n) \cdot 1 < inv(n) \cdot (n \cdot \epsilon)$ (by MultInvariance).
	We have $inv(n) \cdot 1 = inv(n)$ (by One).
	$inv(n) \cdot (n \cdot \epsilon) .= (inv(n) \cdot n) \cdot \epsilon$ (by AssMult)
	$.= 1 \cdot \epsilon$ (by InvDummy)
	$.= \epsilon$ (by OneDummy).
	\end{subproof}
	Hence $dist(a[n],0) = inv(n) < \epsilon$.
	\end{subproof}
	\end{subproof}
	Let us show that $a$ has infinite range.
	\begin{subproof}
	Assume the contrary.
	Take $N$ such that $ran(a) = ranN(a,N)$ (by FiniteRange).
	Then $a[N + 1]$ is an element of $ran(a)$ (by OneNat, AddClosedNat, Range).
	Let us show that $a[N + 1]$ is not an element of $ranN(a,N)$.
	\begin{subproof}
	Let us show that for every natural number $n$ such that $n \leq N \ a[n] \neq a[N + 1]$.
	\begin{subproof}
	Assume the contrary.
	Take $n$ such that $n \leq N \text{ and } a[n] = a[N + 1]$.
	Case $n = 0$.
	We have $2 = inv(N + 1)$.
	$(2 \cdot N) + 2 .= (2 \cdot N) + (2 \cdot 1)$ (by One)
	$.= 2 \cdot (N + 1)$ (by Distrib)
	$.= inv(N + 1) \cdot (N + 1)$
	$.= 1$ (by InvDummy).
	We have $(2 \cdot N) + 2 > 1$.
	Contradiction.
	end.
	Case $n \neq 0$.
	We have $inv(n) = inv(N + 1)$.
	Then $inv(inv(n)) = inv(inv(N + 1))$.
	We have $N + 1 \neq 0$.
	Hence $n = N + 1$ (by InvRule1).
	Contradiction.
	end.
	\end{subproof}
	Hence $a[N + 1]$ is not an element of $ranN(a,N)$ (by RangeN).
	\end{subproof}
	Contradiction.
	\end{subproof}
	\end{proof}
	
	\begin{signature}[Parity]
	$n$ is even is an atom.
	\end{signature}
	
	\noindent Let $n$ is odd stand for not ($n$ is even).
	
	\begin{axiom}[ParityPlusOne]
	$n$ is even iff $n + 1$ is odd.
	\end{axiom}

	\begin{axiom}[ZeroEven]
	$0$ is even.
	\end{axiom}
	
	\begin{lemma}[OneOdd]
	$1$ is odd.
	\end{lemma} 
	
	\begin{lemma}
	Let $a$ be a sequence such that for every $n \ (((n \text{ is even}) \Rightarrow a[n] = 1) \text{ and } ((\text{n is odd}) \Rightarrow a[n] = -1))$.
	Then $a$ diverges and $a$ has finite range.
	\end{lemma}
	\begin{proof}
	Let us show that $a$ diverges.
	\begin{subproof}
	Assume the contrary.
	Take a real number $x$ such that $a$ converges to $x$.
	Take $N$ such that for every $n$ such that $N < n \ dist(a[n],x) < 1$ (by Convergence, OnePos).
	Let us show that $dist(a[N + 1],a[N + 2]) = 2$.
	\begin{subproof}
	Case $N + 1$ is even.
	Then $N + 2$ is odd.
	Hence $dist(a[N + 1],a[N + 2]) = dist(1,-1) = 2$.
	end.
	Case $N + 1$ is odd.
	Then $N + 2$ is even.
	Hence $dist(a[N + 1],a[N + 2]) = dist(-1,1) = 2$.
	end.
	\end{subproof}
	$a[N + 1]$ is a real number and $a[N + 2]$ is a real number.
	We have $dist(a[N + 1],a[N + 2]) \leq dist(a[N + 1],x) + dist(x,a[N + 2])$ (by DistTriangleIneq).
	We have $dist(x,a[N + 2]) = dist(a[N + 2],x)$ (by DistSymm).
	Hence $dist(a[N + 1],a[N + 2]) \leq dist(a[N + 1],x) + dist(a[N + 2],x)$.
	We have $dist(a[N + 1],x) < 1$ and $dist(a[N + 2],x) < 1$.
	Hence $dist(a[N + 1],x) + dist(a[N + 2],x) < 1 + 1$ (by AddInvariance).
	Hence $dist(a[N + 1],a[N + 2]) < 2$ (by MixedTransitivity).
	Hence $2 < 2$.
	Contradiction.
	\end{subproof}
	Let us show that $a$ has finite range.
	\begin{subproof}
	Let us show that $ran(a) = ranN(a,1)$.
	\begin{subproof}
	Let us show that every element of $ranN(a,1)$ is an element of $ran(a)$.
	\begin{subproof}
	Assume $x$ is an element of $ranN(a,1)$.
	Take $n$ such that $n \leq 1$ and $a[n] = x$ (by OneNat, RangeN).
	Hence $x$ is an element of $ran(a)$ (by Range).
	\end{subproof}
	Let us show that every element of $ran(a)$ is an element of $ranN(a,1)$.
	\begin{subproof}
	Assume $x$ is an element of $ran(a)$.
	Let us show that $x = 1$ or $x = -1$.
	\begin{subproof}
	Take $n$ such that $a[n] = x$ (by Range).
	$n$ is even or $n$ is odd.
	Hence $x = 1$ or $x = -1$.
	\end{subproof}
	We have $a[0] = 1$.
	We have $a[1] = -1$.
	Case $x = 1$.
	Then $x = a[0]$.
	We have $0 \leq 1$.
	Hence $x$ is an element of $ranN(a,1)$ (by ZeroNat, OneNat, RangeN).
	end.
	Case $x = -1$.
	Then $x = a[1]$. 
	We have $1 \leq 1$.
	Hence $x$ is an element of $ranN(a,1)$ (by OneNat, RangeN).
	end.
	\end{subproof}
	\end{subproof}
	\end{subproof}
	\end{proof}
	
	\begin{definition}[Neighb]
	Let $\epsilon$ be a positive real number. Let $x$ be a real number.
	$Neighb(x,\epsilon) = \{y \mid y \text{ is a real number such that } dist(y,x) < \epsilon\}$.
	\end{definition}
	
	\begin{theorem}[ConvNeighborhood]
	Let $a$ be a sequence. Let $x$ be a real number.
	$a$ converges to $x$ iff for every positive real number $\epsilon$ there exists a $N$
	such that for every $n$ such that $N < n \ a[n]$ is an element of $Neighb(x,\epsilon)$.
	\end{theorem}
	\begin{proof}
	Let us show that (if $a$ converges to $x$ then for every positive real number $\epsilon$ there exists a $N$
	such that for every $n$ such that $N < n \ a[n]$ is an element of $Neighb(x,\epsilon)$).
	\begin{subproof}
	Assume $a$ converges to $x$.
	Let $\epsilon$ be a positive real number.
	Take $N$ such that for every $n$ such that $N < n \ dist(a[n],x) < \epsilon$ (by Convergence).
	Hence for every $n$ such that $N < n \ a[n]$ is an element of $Neighb(x,\epsilon)$ (by Neighb).
	\end{subproof}
	Let us show that (if for every positive real number $\epsilon$ there exists a $N$ such that
	for every $n$ such that $N < n \ a[n]$ is an element of $Neighb(x,\epsilon)$ then $a$ converges to $x$).
	\begin{subproof}
	Assume for every positive real number $\epsilon$ there exists a $N$ such that
	for every $n$ such that $N < n \ a[n]$ is an element of $Neighb(x,\epsilon)$.
	Let $\epsilon$ be a positive real number.
	Take $N$ such that for every $n$ such that $N < n \ a[n]$ is an element of $Neighb(x,\epsilon)$.
	Hence for every $n$ such that $N < n \ dist(a[n],x) < \epsilon$ (by Neighb).
	\end{subproof}
	\end{proof}	
	
	\begin{lemma}[DistEqual]
	Let $x$ and $y$ be real numbers. Assume for every positive real number $\epsilon$ $dist(x,y) < \epsilon$.
	Then $x = y$.
	\end{lemma}
	\begin{proof}
	Assume the contrary.
	Then there exists a positive real number $\epsilon2$ such that $\epsilon2 < dist(x,y)$.
	A contradiction.
	\end{proof}
	
	\begin{theorem}[LimitUnique]
	Let $a$ be a sequence. Let $x, y$ be real numbers. Assume $a$ converges to $x$ and $a$ converges to $y$.
	Then $x = y$.
	\end{theorem}
	\begin{proof}
	Let us show that for every positive real number $\epsilon \ dist(x,y) < \epsilon$.
	\begin{subproof}
	Let $\epsilon$ be a positive real number.
	Take a positive real number $\halfeps$ such that $\halfeps = inv(2) \cdot \epsilon$.
	Take $N1$ such that for every $n$ such that $N1 < n \ dist(a[n],x) < \halfeps$ (by Convergence).
	Take $N2$ such that for every $n$ such that $N2 < n \ dist(a[n],y) < \halfeps$ (by Convergence).
	For every $n \ dist(x,y) \leq dist(x,a[n]) + dist(a[n],y)$ (by DistTriangleIneq). 
	Take $N3$ such that $N3 = max(N1,N2) + 1$.
	Then $N1 < N3$ and $N2 < N3$.
	Hence $dist(x,a[N3]) < \halfeps$ and $dist(a[N3],y) < \halfeps$ (by DistSymm).
	Hence $dist(x,a[N3]) + dist(a[N3],y) < \halfeps + \halfeps$ (by AddInvariance).
	Hence $dist(x,y) < \halfeps + \halfeps$ (by MixedTransitivity).
	Hence $dist(x,y) < \epsilon$ (by TwoHalf).
	\end{subproof}
	Therefore $x = y$ (by DistEqual).
	\end{proof}
	
	\begin{definition}[BoundedBy]
	Let $a$ be a sequence. Let $K$ be a real number. $a$ is bounded by $K$ iff
	for every $n \ abs(a[n]) \leq K$.
	\end{definition}
	
	\begin{definition}[BoundedSequence]
	Let $a$ be a sequence. $a$ is bounded iff there exists a real number $K$ such that
	$a$ is bounded by $K$.
	\end{definition}
	
	\begin{signature}[MaxN]
	Let $a$ be a sequence. $maxN(a,N)$ is a real number such that
	(there exists $n$ such that $n \leq N$ and $maxN(a,N) = a[n]$) and
	(for every $n$ such that $n \leq N \ a[n] \leq maxN(a,N)$).
	\end{signature}
	
	\begin{theorem}[ConvergentImpBounded]
	Let $a$ be a sequence. Assume that $a$ converges. Then $a$ is bounded.
	\end{theorem}
	\begin{proof}
	Take a real number $x$ such that $a$ converges to $x$.
	Take $N$ such that for every $n$ such that $N < n \ dist(a[n],x) < 1$ (by Convergence, OnePos).
	Define $b[k] = abs(a[k])$ for $k$ in $\NN$.
	Take a real number $K$ such that $K = max(1 + abs(x), maxN(b,N))$.
	Let us show that $a$ is bounded by $K$.
	\begin{subproof}
	Let us show that for every $n \ abs(a[n]) \leq K$.
	\begin{subproof} 
	Let $n$ be a natural number.
	We have $n \leq N$ or $n > N$.
	Case $n \leq N$.
	We have $abs(a[n]) = b[n] \leq maxN(b,N)$ (by MaxN).
	We have $maxN(b,N) \leq K$ (by MaxIneqDummy).
	Therefore $abs(a[n]) \leq K$ (by LeqTransitivity).
	end.
	Case $n > N$.
	We have $dist(a[n],x) < 1$.
	We have $1 + abs(x) \leq K$ (by MaxIneq).
	$abs(a[n]) .= abs(a[n] + 0)$ (by Zero)
	$.= abs(a[n] + (x - x))$ (by Neg)
	$.= abs(a[n] + ((-x) + x))$ (by ComAdd)
	$.= abs((a[n] - x) + x)$ (by AssAdd).
	Hence $abs(a[n]) \leq abs(a[n] - x) + abs(x)$ (by AbsTriangleIneq).
	Hence $abs(a[n]) \leq dist(a[n],x) + abs(x)$.
	We have $dist(a[n],x) + abs(x) < 1 + abs(x)$ (by MixedAddInvariance).
	Hence $abs(a[n]) \leq 1 + abs(x)$ (by MixedTransitivity).
	Therefore $abs(a[n]) \leq K$ (by LeqTransitivity).
	end.
	\end{subproof}
	Hence $a$ is bounded by $K$ (by BoundedBy).
	\end{subproof}
	\end{proof}
	
	\begin{definition}[LimitPointOfSet]
	Let $E$ be a set. Assume every element of $E$ is a real number. A limit point of $E$
	is a real number $x$ such that for every positive real number $\epsilon$ there exists an element
	$y$ of $E$ such that $y$ is an element of $Neighb(x,\epsilon)$ and $y \neq x$.
	\end{definition}
	
	\begin{theorem}[ConvLimitPoint]
	Let $E$ be a set. Assume every element of $E$ is a real number. Let $x$ be a limit point of $E$.
	Then there exists a sequence $a$ such that $a$ converges to $x$ and for every $n \ a[n]$ is an element of $E$.
	\end{theorem}
	\begin{proof}
	Let us show that for every $n$ such that $n > 0$ there exists an element $y$ of $E$ such that
	$y$ is an element of $Neighb(x,inv(n))$ and $y \neq x$.
	\begin{subproof}
	Assume $n > 0$.
	Then $inv(n)$ is a positive real number.
	Take an element $y$ of $E$ such that $y$ is an element of $Neighb(x,inv(n))$
	and $y \neq x$ (by LimitPointOfSet).
	\end{subproof}
	Define $a[n] =$ Case $n = 0 \rightarrow$ Choose an element $y$ of $E$ such that $y$ is an element of
	$Neighb(x,1)$ and $y \neq x$ in $y$,
	Case $n > 0 \rightarrow$ Choose an element $y$ of $E$ such that $y$ is an element of
	$Neighb(x,inv(n))$ and $y \neq x$ in $y$
	for $n$ in $\NN$.
	$a$ is a sequence.	
	Then for every $n \ a[n]$ is an element of $E$.
	Let us show that $a$ converges to $x$.
	\begin{subproof}
	Let $\epsilon$ be a positive real number.
	Take $N$ such that $1 < N \cdot \epsilon$ (by ArchimedeanAxiom, OnePos).
	Let us show that for every $n$ such that $N < n \ dist(a[n],x) < \epsilon$.
	\begin{subproof}
	Assume $N < n$. Then $n \neq 0$.
	Then $a[n]$ is an element of $E$ such that $a[n]$ is an element of $Neighb(x,inv(n))$.
	Hence $dist(a[n],x) < inv(n)$.
	Let us show that $inv(n) < \epsilon$.
	\begin{subproof}
	We have $N \cdot \epsilon < n \cdot \epsilon$ (by ComMult, MultInvariance).
	Hence $1 < n \cdot \epsilon$ (by TransitivityOfOrder).
	$inv(n)$ is positive.
	Hence $inv(n) \cdot 1 < inv(n) \cdot (n \cdot \epsilon)$ (by MultInvariance).
	We have $inv(n) \cdot 1 = inv(n)$ (by One).
	$inv(n) \cdot (n \cdot \epsilon) .= (inv(n) \cdot n) \cdot \epsilon$ (by AssMult)
	$.= 1 \cdot \epsilon$ (by InvDummy)
	$.= \epsilon$ (by OneDummy).
	\end{subproof}
	Hence $dist(a[n],x) < \epsilon$ (by TransitivityOfOrder).
	\end{subproof}
	\end{subproof}
	\end{proof}
	
	\begin{definition}[SequenceSum]
	Let $a,b$ be sequences. $a \plusone b$ is a sequence such that for every $n \ (a \plusone b)[n] = a[n] + b[n]$.
	\end{definition}
	
	\begin{definition}[SequenceProd]
	Let $a,b$ be sequences. $a \cdotone b$ is a sequence such that for every $n \ (a \cdotone b)[n] = a[n] \cdot b[n]$.
	\end{definition}
	
	\begin{definition}[SequenceConstSum]
	Let $a$ be a sequence. Let $c$ be a real number. $c \plustwo a$ is a sequence such that for every $n \ (c \plustwo a)[n] = c + a[n]$.
	\end{definition}
	
	\begin{definition}[SequenceConstProd]
	Let $a$ be a sequence. Let $c$ be a real number. $c \cdottwo a$ is a sequence such that for every $n \ (c \cdottwo a)[n] = c \cdot a[n]$.
	\end{definition}

	\begin{definition}[SequenceDiv]
	Let $a$ be a sequence. Assume for every $n \ a[n] \neq 0$. $div(a)$ is a sequence such that for every $n \ (div(a))[n] = inv(a[n])$.
	\end{definition}
	
	\begin{theorem}[SumConv]
	Let $a,b$ be sequences. Let $x,y$ be real numbers. Assume $a$ converges to $x$ and $b$ converges to $y$.
	Then $a \plusone b$ converges to $x + y$.
	\end{theorem}
	\begin{proof}
	Let $\epsilon$ be a positive real number.
	Take a positive real number $\halfeps$ such that $\halfeps = inv(2) \cdot \epsilon$.
	Take $N1$ such that for every $n$ such that $N1 < n \ dist(a[n],x) < \halfeps$ (by Convergence).
	Take $N2$ such that for every $n$ such that $N2 < n \ dist(b[n],y) < \halfeps$ (by Convergence).
	Take $N$ such that $N = max(N1,N2)$.
	Then $N1 \leq N$ and $N2 \leq N$.
	Let us show that for every $n$ such that $N < n \ dist((a \plusone b)[n],(x+y)) < \epsilon$.
	\begin{subproof}
	Assume $N < n$.
	We have $dist(a[n],x) < \halfeps$.
	We have $dist(b[n],y) < \halfeps$.
	$abs((a[n] + b[n]) - (x + y)) .= abs((a[n] + b[n]) + ((-x) + (-y)))$ (by MinusRule1)
	$.= abs((-x) + ((a[n] + b[n]) - y))$ (by BubbleAdd)
	$.= abs((-x) + (a[n] + (b[n] - y)))$ (by AssAdd)
	$.= abs(((-x) + a[n]) + (b[n] - y))$ (by AssAdd)
	$.= abs((a[n] - x) + (b[n] - y))$ (by ComAdd).
	We have $abs((a[n] - x) + (b[n] - y)) \leq abs(a[n] - x) + abs(b[n] - y)$  (by AbsTriangleIneq).
	Hence $abs((a[n] + b[n]) - (x + y)) \leq dist(a[n],x) + dist(b[n],y)$.
	Hence $dist(a[n],x) + dist(b[n],y) < \halfeps + \halfeps$ (by AddInvariance).
	Hence $abs((a[n] + b[n]) - (x + y)) < \halfeps + \halfeps$ (by MixedTransitivity).
	Hence $dist((a \plusone b)[n],(x + y)) < \halfeps + \halfeps$.
	Hence $dist((a \plusone b)[n],(x + y)) < \epsilon$ (by TwoHalf).
	\end{subproof}
	\end{proof}
	
	\begin{lemma}[ConstConv]
	Let $c$ be a real number. Let $cn$ be a sequence such that for every $n \ cn[n] = c$.
	Then $cn$ converges to $c$.
	\end{lemma}
	\begin{proof}
	Let $\epsilon$ be a positive real number.
	Let us show that for every $n$ such that $0 < n \ dist(cn[n],c) < \epsilon$.
	\begin{subproof}
	Assume $0 < n$.
	$dist(cn[n],c) = dist(c,c) = 0$ (by IdOfInd).
	Hence $dist(cn[n],c) < \epsilon$.
	\end{subproof}
	\end{proof}
	
	\begin{theorem}[SumConstConv]
	Let $a$ be a sequence. Let $x,c$ be real numbers. Assume $a$ converges to $x$.
	Then $c \plustwo a$ converges to $c + x$.
	\end{theorem}
	\begin{proof}
	Define $cn[n] = c$ for $n$ in $\NN$.
	$cn$ is a sequence.
	Let us show that $c \plustwo a = (cn \plusone a)$.
	\begin{subproof}
	Let us show that for every $n \ (c \plustwo a)[n] = (cn \plusone a)[n]$.
	\begin{subproof}
	Let $n$ be a natural number.
	$(c \plustwo a)[n] .= c + a[n]
	.= cn[n] + a[n]
	.= (cn \plusone a)[n]$.
	\end{subproof}
	Hence $c \plustwo a = (cn \plusone a)$ (by SequenceEq).
	\end{subproof}
	$cn$ converges to $c$ (by ConstConv).
	Then $c \plustwo a$ converges to $c + x$ (by SumConv).
	\end{proof}
	
	\begin{theorem}[ProdConstConv]
	Let $a$ be a sequence. Let $x,c$ be real numbers. Assume $a$ converges to $x$.
	Then $c \cdottwo a$ converges to $c \cdot x$.
	\end{theorem}
	\begin{proof}
	Case $c = 0$.
	We have $c \cdot x = 0$.
	Let us show that for every $n \ (c \cdottwo a)[n] = 0$. 
	\begin{subproof}
	$(c \cdottwo a)[n] .= c \cdot a[n]
	.= 0 \cdot a[n]
	.= 0$ (by ZeroMult).
	\end{subproof}
	Hence $c \cdottwo a$ converges to $c \cdot x$ (by ConstConv).
	end.
	Case $c \neq 0$.
	Let $\epsilon$ be a positive real number. 
	(1)     Take a positive real number $\inveps$ such that $\inveps = inv(abs(c)) \cdot \epsilon$.
	Take $N$ such that for every $n$ such that $N < n \ dist(a[n],x) < \inveps$ (By Convergence).
	Let us show that for every $n$ such that $N < n \ dist(c \cdot a[n],c \cdot x) < \epsilon$.
	\begin{subproof}
	Assume $N < n$.
	$abs((c \cdot a[n]) - (c \cdot x)) .= abs((c \cdot a[n]) +  ((-1) \cdot (c \cdot x))) (by MinusRule4)$ 
	$.= abs((c \cdot a[n]) + (((-1) \cdot c) \cdot x))$ (by AssMult)
	$.= abs((c \cdot a[n]) + ((c \cdot (-1)) \cdot x))$ (by ComMult)
	$.= abs((c \cdot a[n]) + (c \cdot ((-1) \cdot x)))$ (by AssMult)
	$.= abs((c \cdot a[n]) + (c \cdot (-x)))$ (by MinusRule4)
	$.= abs(c \cdot (a[n] - x))$ (by Distrib)
	$.= abs(c) \cdot abs(a[n] - x)$ (by AbsMult).
	$abs(c) \cdot dist(a[n],x) < abs(c) \cdot \inveps$ (by AbsPos, MultInvariance).
	Hence $abs((c \cdot a[n]) - (c \cdot x)) < abs(c) \cdot \inveps$.
	$abs(c) \cdot \inveps .= abs(c) \cdot (inv(abs(c)) \cdot \epsilon)$ (by 1)
	$.= (abs(c) \cdot inv(abs(c))) \cdot \epsilon$ (by AssMult)
	$.= 1 \cdot \epsilon$ (by Inverse)
	$.= \epsilon$ (by OneDummy).
	Hence $dist(c \cdot a[n],c \cdot x) < \epsilon$.
	\end{subproof}
	end.
	\end{proof}

	\begin{lemma}[ConstMultSum]
	Let $a,b$ be sequences. Let $x,y$ be real numbers such that for every $n\ b[n] = y \cdot (a[n] + (-x))$. Assume $a$ converges to $x$.
	Then $b$ converges to $0$.
	\end{lemma}
	\begin{proof}
	Define $sum[k] = (-x) + a[k]$ for $k$ in $\NN$.
	$sum$ is a sequence.
	Let us show that $sum$ converges to $0$.
	\begin{subproof}
	We have $sum = (-x) \plustwo a$.
	Hence $sum$ converges to $(-x) + x$ (by SumConstConv).
	$(-x) + x = 0$ (by ComAdd, Neg).
	\end{subproof}
	Let us show that for every $n\ b[n] = y \cdot sum[n]$.
	\begin{subproof}
	$b[n] .= y \cdot (a[n] + (-x))$
	$.= y \cdot ((-x) + a[n])$ (by ComAdd)
	$.= y \cdot sum[n]$.
	\end{subproof}
	Hence $b = y \cdottwo sum$ (by SequenceEq).
	Hence $b$ converges to $0$ (by ProdConstConv, ComMult, ZeroMult).
	\end{proof}
	
	\begin{theorem}[ProdConv]
	Let $a,b$ be sequences. Let $x,y$ be real numbers. Assume $a$ converges to $x$ and $b$ converges to $y$.
	Let $a \cdotone b$ be a sequence such that for every natural number $n (a \cdotone b)[n] = a[n] \cdot b[n]$.
	Then $a \cdotone b$ converges to $x \cdot y$.
	\end{theorem}
	\begin{proof}
	(1) Define $s1[k] = (a[k] - x) \cdot (b[k] - y)$ for $k$ in $\NN$.
	Let us show that $s1$ converges to $0$. 
	\begin{subproof}
    Assume $\epsilon$ is a positive real number. 
    Take a positive real number $\rooteps$ such that $\rooteps = sqrt(\epsilon)$ (by Sqrt).
    Take $N1$ such that for every $n$ such that $N1 < n \ dist(a[n],x) < \rooteps$ (by Convergence).
    Take $N2$ such that for every $n$ such that $N2 < n \ dist(b[n],y) < \rooteps$ (by Convergence).
    Take $N$ such that $N = max(N1,N2)$.
    Let us show that for every $n$ such that $N < n \ dist(s1[n],0) < \epsilon$.
    \begin{subproof}
    Assume $N < n$.
    $dist(a[n],x) < \rooteps$ and $dist(b[n],y) < \rooteps$.
    $dist(a[n],x), dist(b[n],y)$ and $\rooteps$ are nonnegative.
    Then $dist(a[n],x) \cdot dist(b[n],y) < \epsilon$ (by NonNegMultInvariance).
    Hence $abs(a[n] - x) \cdot abs(b[n] - y) < \epsilon$.
    Hence $abs((a[n] - x) \cdot (b[n] - y)) < \epsilon$ (by AbsMult).
    Hence $abs(((a[n] - x) \cdot (b[n] - y)) - 0) < \epsilon$ (by Zero, NegOfZero).
    \end{subproof}
	\end{subproof}
	(2) Define $s2[k] = (x \cdot (b[k] + (-y))) + (y \cdot (a[k] + (-x)))$ for $k$ in $\NN$.
	Let us show that $s2$ converges to $0$.
	\begin{subproof}
	(3) Define $s2a[k] = y \cdot (a[k] + (-x))$ for $k$ in $\NN$.
	(4) Define $s2b[k] = x \cdot (b[k] + (-y))$ for $k$ in $\NN$.
	$s2a, s2b$ are sequences.
	Define $sum1[k] = a[k] + (-x)$ for $k$ in $\NN$.
	Define $sum2[k] = b[k] + (-y)$ for $k$ in $\NN$.
	$sum1, sum2$ are sequences.
	$sum1 = (-x) \plustwo a$ and $sum2 = (-y) \plustwo b$ (by ComAdd, SequenceEq).
	$sum1, sum2$ converge to $0$ (by SumConstConv, Neg, ComAdd).
	We have $s2a = y \cdottwo sum1$ and $s2b = x \cdottwo sum2$ (by SequenceEq).
	$s2a, s2b$ converge to $0$ (by ConstMultSum). 
	Let us show that for every $n \ s2[n] = s2b[n] + s2a[n]$.
	\begin{subproof}
	$s2[n] .= (x \cdot (b[n] + (-y))) + (y \cdot (a[n] + (-x)))$ (by 2)
	$.= s2b[n] + s2a[n]$ (by 3, 4).
	\end{subproof}
	Hence $s2$ converges to $0$ (by SumConv).
	\end{subproof}
	(3) Define $s3[k] = (a[k] \cdot b[k]) - (x \cdot y)$ for $k$ in $\NN$.
	Let us show that $s3$ converges to $0$.
	\begin{subproof}
	Let us show that for every $n \ s3[n] = s1[n] + s2[n]$.
	\begin{subproof}
	$s3[n] .= (a[n] \cdot b[n]) - (x \cdot y)$ (by 3)
	$.= ((a[n] - x) \cdot (b[n] - y)) + ((x \cdot (b[n] - y)) + (y \cdot (a[n] - x)))$ (by Identity1)
	$.= s1[n] + s2[n]$ (by 1, 2).
	\end{subproof}
	Hence $s3 = s1 \plusone s2$ (by SequenceEq).
	Therefore the thesis (by SumConv, Zero).
	\end{subproof}
	Let $\epsilon$ be a positive real number.
	Take $N$ such that for every $n$ such that $N < n \ dist(s3[n],0) < \epsilon$ (by Convergence).
	Let us show that for every $n$ such that $N < n \ dist(a[n] \cdot b[n],x \cdot y) < \epsilon$.
	\begin{subproof}
	Assume $N < n$.
	$dist(s3[n],0) .= dist((a[n] \cdot b[n]) - (x \cdot y),0)$ (by 3)
	$.= abs(((a[n] \cdot b[n]) - (x \cdot y)) - 0)$ (by DistDefinition)
	$.= abs((a[n] \cdot b[n]) - (x \cdot y))$ (by NegOfZero, Zero)
	$.= dist(a[n] \cdot b[n],x \cdot y)$ (by DistDefinition).
	\end{subproof}
	\end{proof}

	\begin{theorem}[DivConv]
	Let $a$ be a sequence. Let $x$ be a real number such that $x \neq 0$. Assume $a$ converges to $x$. 
	Assume for every $n \ a[n] \neq 0$.
	Let $div(a)$ be a sequence such that for every natural number $n \ (div(a))[n] = inv(a[n])$.
	Then $div(a)$ converges to $inv(x)$.
	\end{theorem}
	\begin{proof}
	Let $\epsilon$ be a positive real number.
	$abs(x) \neq 0$. 
	$2, inv(2), abs(x), abs(x) \cdot abs(x), -abs(x), ((-1) \cdot inv(2)) \cdot abs(x), (inv(2) \cdot abs(x)) + (-abs(x))$ are real numbers.
	We have $pos(2)$ and $pos(inv(2))$ and $pos(abs(x))$ and $pos(abs(inv(x)))$ and $pos(inv(2) \cdot \epsilon)$ and $pos(abs(x) \cdot abs(x))$ and $pos((inv(2) \cdot \epsilon) \cdot (abs(x) \cdot abs(x)))$ and
	$pos(inv(2) \cdot abs(x))$ and $pos(\epsilon \cdot (abs(x) \cdot abs(x)))$ and $pos(inv(2) \cdot abs(inv(x)))$ and $pos((\epsilon \cdot (abs(x) \cdot abs(x))) \cdot (inv(2) \cdot abs(inv(x))))$.
	Take a natural number $m$ such that for every $n$ such that $m < n \  dist(a[n],x) < inv(2) \cdot abs(x)$ (by Convergence).
	Let us show that for every $n$ such that $m < n \ inv(2) \cdot abs(x) < abs(a[n])$.
	\begin{subproof}
	Assume $m < n$.
	$a[n], abs(a[n]), -abs(a[n]), abs(x) - abs(a[n]), x - a[n], abs(x - a[n]), a[n] - x, abs(a[n] - x), abs(x) + (-abs(a[n])), (abs(x) + (-abs(a[n]))) + (-abs(x))$ are real numbers.
	Let us show that $abs(x) - abs(a[n]) < inv(2) \cdot abs(x)$.
	\begin{subproof}
	$abs(x) - abs(a[n]) \leq abs(x - a[n])$ (by AbsTriangleIneq2).
	$abs(x - a[n]) = abs(-(x - a[n])) = abs(a[n] - x)$ (by AbsPosNeg, MinusRule1, MinusRule2, ComAdd).
	$abs(a[n] - x) < inv(2) \cdot abs(x)$ (by DistDefinition).
	Hence the thesis (by MixedTransitivity).
	\end{subproof}
	Let us show that $-abs(a[n]) < (-inv(2)) \cdot abs(x)$.
	\begin{subproof}
	$(abs(x) - abs(a[n])) + (-abs(x)) < (inv(2) \cdot abs(x)) + (-abs(x))$ (by MixedAddInvariance). 
	$(abs(x) - abs(a[n])) + (-abs(x)) = -abs(a[n])$ (by AssAdd, Neg, ComAdd, Zero).
	$-abs(a[n]) < (inv(2) \cdot abs(x)) + (-abs(x))$ (by AssAdd, Neg, Zero).
	$(inv(2) \cdot abs(x)) + (-abs(x)) .= (inv(2) \cdot abs(x)) + ((-1) \cdot abs(x))$ (by MinusRule4)
	$.= (inv(2) + (-1)) \cdot abs(x)$ (by DistribDummy)
	$.= ((1 \cdot inv(2)) + (-(1 \cdot inv(1)))) \cdot abs(x)$ (by OneDummy, Inverse)
	$.= ((1 \cdot inv(2)) + ((-1) \cdot inv(1))) \cdot abs(x)$ (by OneDummy, MinusRule4)
	$.= ((1 \cdot inv(2)) + ((-1) \cdot inv(1))) \cdot abs(x)$ (by MinusRule4)
	$.= (((1 \cdot 1) + (2 \cdot (-1))) \cdot inv(2 \cdot 1)) \cdot abs(x)$ (by InvAdd)
	$.= ((1 + ((-1) \cdot 2)) \cdot inv(2)) \cdot abs(x)$ (by One, ComMult).
	$1 + ((-1) \cdot 2) .= 1 + -(1 \cdot 2)$ (by MinusRule5)
	$.= -(2 - 1)$ (by OneDummy, MinusRule3)
	$.= -1$. 
	\end{subproof}
    Therefore $abs(a[n]) > -((-inv(2)) \cdot abs(x))$ (by OrdMirror, MinusRule2).
    $-((-inv(2)) \cdot abs(x)) = abs(x) \cdot inv(2)$ (by MinusRule5, MinusRule2, ComMult).
    Therefore $abs(a[n]) > abs(x) \cdot inv(2)$ (by TransitivityOfOrder).
	\end{subproof}
	Take $N1$ such that for every $n$ such that $N1 < n \ dist(a[n],x) < (inv(2) \cdot \epsilon) \cdot (abs(x) \cdot abs(x))$ (by Convergence). 
	Take $N2$ such that $N2 = max(N1,m)$.
	Let us show that for every $n$ such that $N2 < n \ dist(inv(a[n]),inv(x)) < \epsilon$.
	\begin{subproof}
	Assume $N2 < n$.
	Then we have $N1 < n$ and $m < n$.
	Let us show that $dist(inv(a[n]),inv(x)) < ((\epsilon \cdot (abs(x) \cdot abs(x))) \cdot (inv(2) \cdot abs(inv(x)))) \cdot (1 \cdot inv(abs(a[n])))$.
	\begin{subproof}
	$dist(inv(a[n]),inv(x)) .= abs(inv(a[n]) - inv(x))$ (by DistDefinition)
	$.= abs((1 \cdot inv(a[n])) - (1 \cdot inv(x)))$ (by OneDummy)
	$.= abs((1 \cdot inv(a[n])) + ((-1) \cdot inv(x)))$ (by MinusRule5)
	$.= abs(((1 \cdot x) + (a[n] \cdot (-1))) \cdot inv(a[n] \cdot x))$ (by InvAdd)
	$.= abs((x + ((-1) \cdot a[n])) \cdot inv(a[n] \cdot x))$ (by OneDummy, ComMult)
	$.= abs((x - a[n]) \cdot inv(a[n] \cdot x))$ (by MinusRule4)
	$.= abs(x - a[n]) \cdot abs(inv(a[n]) \cdot inv(x))$ (by AbsMult, InvRule2)
	$.= abs(inv(a[n]) \cdot inv(x)) \cdot abs(x - a[n])$ (by ComMult).
	We have $pos(abs(inv(a[n]) \cdot inv(x)))$ (by AbsPos, InvNotZero, NoZeroDivisors).
	$abs(x - a[n]) = dist(a[n],x)$ (by DistDefinition, DistSymm).
	$abs(inv(a[n]) \cdot inv(x)) \cdot abs(x - a[n]) < abs(inv(a[n]) \cdot inv(x)) \cdot ((inv(2) \cdot \epsilon) \cdot (abs(x) \cdot abs(x)))$ (by MultInvariance, DistDefinition).
	$abs(inv(a[n]) \cdot inv(x)) \cdot ((inv(2) \cdot \epsilon) \cdot (abs(x) \cdot abs(x))) .= (abs(inv(a[n])) \cdot abs(inv(x))) \cdot ((inv(2) \cdot \epsilon) \cdot (abs(x) \cdot abs(x)))$ (by AbsMult)
	$.= ((inv(2) \cdot \epsilon) \cdot (abs(x) \cdot abs(x))) \cdot (abs(inv(a[n])) \cdot abs(inv(x)))$ (by ComMult)
	$.= (inv(2) \cdot (\epsilon \cdot (abs(x) \cdot abs(x)))) \cdot (abs(inv(a[n])) \cdot abs(inv(x)))$ (by AssMult)
	$.= ((\epsilon \cdot (abs(x) \cdot abs(x))) \cdot inv(2)) \cdot (abs(inv(a[n])) \cdot abs(inv(x)))$ (by ComMult) 
	$.= (\epsilon \cdot (abs(x) \cdot abs(x))) \cdot (inv(2) \cdot (abs(inv(a[n])) \cdot abs(inv(x))))$ (by AssMult)
	$.= (\epsilon \cdot (abs(x) \cdot abs(x))) \cdot (inv(2) \cdot (abs(inv(x)) \cdot abs(inv(a[n]))))$ (by ComMult)
	$.= (\epsilon \cdot (abs(x) \cdot abs(x))) \cdot ((inv(2) \cdot abs(inv(x))) \cdot abs(inv(a[n])))$ (by AssMult)
	$.= ((\epsilon \cdot (abs(x) \cdot abs(x))) \cdot (inv(2) \cdot abs(inv(x)))) \cdot abs(inv(a[n]))$ (by AssMult)
	$.= ((\epsilon \cdot (abs(x) \cdot abs(x))) \cdot (inv(2) \cdot abs(inv(x)))) \cdot inv(abs(a[n]))$ (by AbsInv)
	$.= ((\epsilon \cdot (abs(x) \cdot abs(x))) \cdot (inv(2) \cdot abs(inv(x)))) \cdot (1 \cdot inv(abs(a[n])))$ (by OneDummy).
	\end{subproof}
	$(\epsilon \cdot (abs(x) \cdot abs(x))) \cdot (inv(2) \cdot abs(inv(x))), 1 \cdot inv(abs(a[n])), 2 \cdot inv(abs(x)), dist(inv(a[n]), inv(x)),
	((\epsilon \cdot (abs(x) \cdot abs(x))) \cdot (inv(2) \cdot abs(inv(x)))) \cdot (1 \cdot inv(abs(a[n]))), ((\epsilon \cdot (abs(x) \cdot abs(x))) \cdot (inv(2) \cdot abs(inv(x)))) \cdot (2 \cdot inv(abs(x)))$ are real numbers.
	Let us show that $((\epsilon \cdot (abs(x) \cdot abs(x))) \cdot (inv(2) \cdot abs(inv(x)))) \cdot (1 \cdot inv(abs(a[n]))) < \epsilon$.
	\begin{subproof} 
	Let us show that $1 \cdot inv(abs(a[n])) < 2 \cdot inv(abs(x))$.
	\begin{subproof}
	We have $abs(a[n]) > abs(x) \cdot inv(2)$.
	Hence $abs(a[n]) \cdot inv(1) > abs(x) \cdot inv(2)$ (by InvOne, One).
	We have ($pos(abs(a[n]))$ and $pos(1)$) and ($pos(abs(x))$ and $pos(2)$).
	($abs(a[n]) \neq 0$ and $1 \neq 0$) and ($abs(x) \neq 0$ and $2 \neq 0$).
	Then $1 \cdot inv(abs(a[n])) < 2 \cdot inv(abs(x))$ (by InvSwapIneq).
	\end{subproof}
	$((\epsilon \cdot (abs(x) \cdot abs(x))) \cdot (inv(2) \cdot abs(inv(x)))) \cdot (1 \cdot inv(abs(a[n]))) < ((\epsilon \cdot (abs(x) \cdot abs(x))) \cdot (inv(2) \cdot abs(inv(x)))) \cdot (2 \cdot inv(abs(x)))$ (by MultInvariance).
	$((\epsilon \cdot (abs(x) \cdot abs(x))) \cdot (inv(2) \cdot abs(inv(x)))) \cdot (2 \cdot inv(abs(x))) .= ((\epsilon \cdot (abs(x) \cdot abs(x))) \cdot (inv(abs(x)) \cdot inv(2))) \cdot (2 \cdot inv(abs(x)))$ (by ComMult, AbsInv)
	$.= (((\epsilon \cdot (abs(x) \cdot abs(x))) \cdot inv(abs(x))) \cdot inv(2)) \cdot (2 \cdot inv(abs(x)))$ (by AssMult)
	$.= ((\epsilon \cdot (abs(x) \cdot abs(x))) \cdot inv(abs(x))) \cdot (inv(2) \cdot (2 \cdot inv(abs(x))))$ (by AssMult)
	$.= (\epsilon \cdot ((abs(x) \cdot abs(x)) \cdot inv(abs(x)))) \cdot ((inv(2) \cdot 2) \cdot inv(abs(x)))$ (by AssMult)
	$.= (\epsilon \cdot ((abs(x) \cdot abs(x)) \cdot inv(abs(x)))) \cdot inv(abs(x))$ (by InvDummy, OneDummy)
	$.= (\epsilon \cdot (abs(x) \cdot (abs(x) \cdot inv(abs(x))))) \cdot inv(abs(x))$ (by AssMult)
	$.= (\epsilon \cdot abs(x)) \cdot inv(abs(x))$ (by Inverse, One)
	$.= \epsilon \cdot (abs(x) \cdot inv(abs(x)))$ (by AssMult)
	$.= \epsilon$ (by Inverse, One).
	\end{subproof}
	Hence the thesis (by TransitivityOfOrder).
	\end{subproof}
	\end{proof}

	
	\begin{definition}[IndexSeq]
		An index sequence is a sequence $i$ such that \\
		(for every $n$ $i[n]$ is a natural number) and (for every $n$ $i[n] < i[n + 1]$).
	\end{definition}
	
	\begin{definition}[SubSeq]
		Let $a$ be a sequence and $i$ be an index sequence. $Subseq(a,i)$ is a sequence such that for every $n$
		$Subseq(a,i)[n] = a[i[n]]$.
	\end{definition}
	
	\begin{definition}[ConvSubSeq]
		Let $a$ be a sequence. $a$ has some convergent subsequence iff there exists an index sequence $i$ such that $Subseq(a,i)$ converges.
	\end{definition}
	
	\begin{axiom}[IndSucc]
		$n$ $\lessdot$ $n + 1$. 
	\end{axiom}
	
	\begin{axiom}[IndPrec]
		Assume $n \neq 0$. Then there exists $m$ such that \\ $n = m + 1$.
	\end{axiom}
	
	\begin{axiom}[IndNonNeg]
		$n \lessdot 0$ for no $n$.
	\end{axiom}
	
	\begin{axiom}[IndPlusOne]
		Assume $n < m$. Then $n + 1 \leq m$.
	\end{axiom}
	
	\begin{lemma}[SubSeqLeq]
		Let $a$ be a sequence. Let $i$ be an index sequence. Then for every $n$ $n \leq i[n]$.
	\end{lemma}
	
	\begin{proof}
		We can show by induction that $n \leq i[n]$ for every $n$.
		\begin{subproof}
			Let $n$ be a natural number.\\
			Case $n = 0$. Obvious.\\
			Case $n \neq 0$.\\ 
			\begin{case}
				Take $m$ such that $n = m + 1$. Then $m \leq i[m]$.\\
				We can show by induction that $i[k] + 1 \leq i[k + 1]$ for every $k$. Obvious.\\
			\end{case}
		\end{subproof}
	\end{proof}
	
	\begin{lemma}[LimitSubSeq]
		Let $a$ be a sequence. Let $x$ be a real number. $a$ converges to $x$ iff for every index sequence $i$ $Subseq(a,i)$ converges to $x$. 
	\end{lemma}
	\begin{proof}
		Let us show that if $a$ converges to $x$ then for every index sequence $i$ $Subseq(a,i)$ converges to $x$.
		
		\begin{subproof}
			Assume $a$ converges to $x$. 
			Let $i$ be an index sequence.
			Let $\epsilon$ be a positive real number.
			Take $N$ such that for every $n$ such that $N < n$ $dist(a[n],x) < \epsilon$ (by Convergence).
			
			Let us show that for every $n$ such that $N < n$ \\ $dist(Subseq(a,i)[n],x) < \epsilon$.
			\begin{subproof}
				Let $n$ be a natural number such that $N < n$.\\
				Then $n \leq i[n]$ (by SubSeqLeq).\\
				Hence $N < i[n]$ (by MixedTransitivity).\\
				Hence $dist(Subseq(a,i)[n],x) = dist(a[i[n]],x) < \epsilon$.
			\end{subproof}
		\end{subproof}
		Let us show that if for every index sequence $i$ $Subseq(a,i)$ converges to $x$ then $a$ converges to $x$.
		\begin{subproof}
			Assume for every index sequence $i$ $Subseq(a,i)$ converges to $x$. \\
			Define $i[n] = n$ for $n$ in $\NN$.\\
			$i$ is an index sequence.\\
			$Subseq(a,i)$ converges to $x$. \\
			For every $n$ $a[n] = Subseq(a,i)[n]$. \\
			Hence $a = Subseq(a,i)$ (by SequenceEq). \\
			Hence $a$ converges to $x$.
		\end{subproof}
	\end{proof}
	
	\begin{axiom}[BolzanoWeierstrass]
		Let $a$ be a bounded sequence. Then $a$ has some convergent subsequence. 
	\end{axiom}
	
	\begin{definition}[Cauchy]
		A cauchy sequence is a sequence $a$ such that for every positive real number $\epsilon$ there exists $N$ such that
		for every $n$,$m$ such that ($N < n$ and $N < m$) $dist(a[n],a[m]) < \epsilon$.
	\end{definition}
	
	\begin{lemma}[CauchyBounded]
		Let $a$ be a cauchy sequence. Then $a$ is bounded.
	\end{lemma}
	\begin{proof}
		Take $N$ such that for every $n$,$m$ such that ($N < n$ and $N < m$) $dist(a[n],a[m]) < 1$ (by Cauchy, OnePos). \\
		$N + 1$ is a natural number and $N < N + 1$.\\
		Hence for every $n$ such that $N < n$ $dist(a[n],a[N + 1]) < 1$.\\
		Define $b[k] = abs(a[k])$ for $k$ in $\NN$.\\
		$maxN(b,N)$, $1$, $a[N + 1]$, $abs(a[N + 1])$, $1 + abs(a[N + 1])$ are real numbers.\\
		Take a real number $K$ such that $K = max(1 + abs(a[N + 1]), maxN(b,N))$.\\
		
		\noindent Let us show that $a$ is bounded by $K$.		
		\begin{subproof}
			Let us show that for every $n$ $abs(a[n]) \leq K$. 
			\begin{subproof}
				Let $n$ be a natural number.\\				
				$a[n]$, $abs(a[n])$, $b[n]$, $dist(a[n],a[N + 1])$, $a[n] - a[N + 1]$, $dist(a[n],a[N + 1]) + abs(a[N + 1])$ are real numbers.\\
				We have $n \leq N$ or $n > N$.\\
				Case $n \leq N$.
				\begin{case}
					We have $abs(a[n]) = b[n] \leq maxN(b,N)$ (by MaxN).
					
					We have $maxN(b,N) \leq K$ (by MaxIneqDummy).
					
					Therefore $abs(a[n]) \leq K$ (by LeqTransitivity).
					
				\end{case}	
				
				Case $n > N$.
				
				\begin{subproof}
					We have $dist(a[n],a[N + 1]) < 1$.\\
					We have $1 + abs(a[N + 1]) \leq K$ (by MaxIneq).\\
					$abs(a[n]) .= abs(a[n] + 0)$ (by Zero)\\
					$.= abs(a[n] + (a[N + 1] - a[N + 1]))$ (by Neg)\\
					$.= abs(a[n] + ((-a[N + 1]) + a[N + 1]))$ (by ComAdd)\\
					$.= abs((a[n] - a[N + 1]) + a[N + 1])$ (by AssAdd).\\
					We have $abs((a[n] - a[N + 1]) + a[N + 1]) \leq abs(a[n] - a[N + 1]) + abs(a[N + 1])$ (by AbsTriangleIneq).\\
					Hence $abs(a[n]) \leq abs(a[n] - a[N + 1]) + abs(a[N + 1])$.\\
					Hence $abs(a[n]) \leq dist(a[n],a[N + 1]) + abs(a[N + 1])$.\\
					We have $dist(a[n],a[N + 1]) + abs(a[N + 1]) < 1 + abs(a[N + 1])$ (by MixedAddInvariance).\\
					Hence $abs(a[n]) \leq 1 + abs(a[N + 1])$ (by MixedTransitivity).\\
					Therefore $abs(a[n]) \leq K$ (by LeqTransitivity).
				\end{subproof}
			\end{subproof}
			Hence $a$ is bounded by $K$ (by BoundedBy).
		\end{subproof}
	\end{proof}
	
	\begin{lemma}[CauchyConvSubSeq]
		Let a be a cauchy sequence such that a has some convergent subsequence. Then a converges.
	\end{lemma}
	
	\begin{proof}
		Take a index sequence $i$ such that $Subseq(a,i)$ converges.
		Take a real number $x$ such that $Subseq(a,i)$ converges to $x$.\\
		
		Let us show that $a$ converges to $x$.
		\begin{subproof}
			Let $\epsilon$ be a positive real number.\\			
			Take a positive real number $\halfeps$ such that $\halfeps = inv(2) \cdot \epsilon$.\\
			Take $N1$ such that for every $n$,$m$ such that ($N1 < n$ and $N1 < m$) $dist(a[n],a[m]) < \halfeps$ (by Cauchy).\\
			Take $N2$ such that for every $n$ such that $N2 < n$ $dist(Subseq(a,i)[n],x) < \halfeps$ (by Convergence).\\
			Take $N$ such that $N = max(N1,N2)$.\\
			Then $N1 \leq N$ and $N2 \leq N$.
			
			Let us show that for every $n$ such that $N < n$ $dist(a[n],x) < \epsilon$.
			\begin{subproof}
				Assume $N < n$. Hence $N1 < n$ and $N2 < n$ (by MixedTransitivity).\\
				We have $n \leq i[n]$ (by SubSeqLeq).\\
				Hence $N1 < i[n]$ (by MixedTransitivity).\\
				$a[n]$, $a[i[n]]$, $dist(a[n],a[i[n]])$, $dist(a[n],x), a[n] - a[i[n]], a[i[n]] - x$, $dist(a[n],a[i[n]]) + dist(a[i[n]],x)$ are real numbers.\\
				We have $Subseq(a,i)[n] = a[i[n]]$.\\
				We have $dist(a[n],a[i[n]]) < \halfeps$.\\
				We have $dist(a[i[n]],x) < \halfeps$.\\
				$dist(a[n],x) .= abs(a[n] - x)$ (by DistDefinition)\\
				$.= abs((a[n] + 0) - x)$ (by Zero)\\
				$.= abs((a[n] + (a[i[n]] - a[i[n]])) - x)$ (by Neg)\\
				$.= abs((a[n] + ((-a[i[n]]) + a[i[n]])) - x)$ (by ComAdd)\\
				$.= abs(((a[n] - a[i[n]]) + a[i[n]]) - x)$ (by AssAdd)\\
				$.= abs((a[n] - a[i[n]]) + (a[i[n]] - x))$ (by AssAdd).\\
				We have $abs((a[n] - a[i[n]]) + (a[i[n]] - x)) \leq abs(a[n] - a[i[n]]) + abs(a[i[n]] - x)$ (by AbsTriangleIneq).\\
				Hence $dist(a[n],x) \leq abs(a[n] - a[i[n]]) + abs(a[i[n]] - x)$.\\
				Hence $dist(a[n],x) \leq dist(a[n],a[i[n]]) + dist(a[i[n]],x)$.\\
				We have $dist(a[n],a[i[n]]) + dist(a[i[n]],x) < \halfeps + \halfeps$ (by AddInvariance).\\
				Hence $dist(a[n],x) < \halfeps + \halfeps$ (by MixedTransitivity).\\
				Hence $dist(a[n],x) < \epsilon$ (by TwoHalf).
			\end{subproof}
			
		\end{subproof}
	\end{proof}
	
	\begin{theorem}[RComplete]
		Let $a$ be a sequence. $a$ is a cauchy sequence iff $a$ converges.
	\end{theorem}
	
	\begin{proof}
		Let us show that (If $a$ converges then $a$ is a cauchy sequence).
		
		\begin{subproof}
			Assume $a$ converges.
			
			Take a real number $x$ such that $a$ converges to $x$.
			
			Let $\epsilon$ be a positive real number.
			
			Take a positive real number $\halfeps$ such that $\halfeps = inv(2) \cdot \epsilon$.
			
			Take $N$ such that for every $n$ such that $N < n$ $dist(a[n],x) < \halfeps$ (by Convergence).
			
			Let us show that for every $n$,$m$ such that ($N < n$ and $N < m$) $dist(a[n],a[m]) < \epsilon$.
			
			\begin{subproof}
				Assume $N < n$ and $N < m$.
				
				We have $dist(a[n],x) < \halfeps$.
				
				We have $dist(a[m],x) < \halfeps$.
				
				We have $dist(a[n],a[m]) \leq dist(a[n],x) + dist(x,a[m])$ (by DistTriangleIneq).
				
				Hence $dist(a[n],a[m]) \leq dist(a[n],x) + dist(a[m],x)$ (by DistSymm).
				
				We have $dist(a[n],x) + dist(a[m],x) < \halfeps + \halfeps$ (by AddInvariance).
				
				Hence $dist(a[n],a[m]) < \halfeps + \halfeps$ (by MixedTransitivity).
				
				Hence $dist(a[n],a[m]) < \epsilon$ (by TwoHalf).
				
			\end{subproof}
			
		\end{subproof}
		
		Let us show that (If $a$ is a cauchy sequence then $a$ converges).
		
		\begin{subproof}
			Assume $a$ is a cauchy sequence.
			
			Then $a$ is bounded (by CauchyBounded).
			
			Therefore $a$ has some convergent subsequence (by BolzanoWeierstrass).
			
			Hence $a$ converges (by CauchyConvSubSeq).
			
		\end{subproof}
	\end{proof}
	
	
	\begin{definition}[MonInc]
		Let $a$ be a sequence. $a$ is monotonically increasing iff (for every $n$,$m$ such that $n \leq m$ $a[n] \leq a[m]$).
	\end{definition}
	
	\begin{definition}[MonDec]
		Let $a$ be a sequence. $a$ is monotonically decreasing iff (for every $n$,$m$ such that $n \leq m$ $a[n] \geq a[m]$).
	\end{definition}
	
	\begin{definition}[Mon]
		Let $a$ be a sequence. $a$ is monotonic iff $a$ is monotonically increasing or $a$ is monotonically decreasing.
	\end{definition}
	
	\begin{definition}[UpperBoundSeq]
		Let $a$ be a bounded sequence. Let $K$ be a real number. $K$ is upper bound of $a$ iff (for every $n$ $a[n] \leq K$).
	\end{definition}
	
	\begin{definition}[LeastUpperBoundSeq]
		Let $a$ be a bounded sequence. $LeastUpper(a)$ is a real number $K$ such that ($K$ is upper bound of $a$) and 
		(for every real number $L$ such that $L$ is upper bound of $a$ $K \leq L$).
	\end{definition}
	
	\begin{definition}[LowerBoundSeq]
		Let $a$ be a bounded sequence. Let $K$ be a real number. $K$ is lower bound of $a$ iff (for every $n$ $a[n] \geq K$).
	\end{definition}
	
	\begin{definition}[GreatestLowerBoundSeq]
		Let $a$ be a bounded sequence. $GreatestLower(a)$ is a real number $K$ such that ($K$ is lower bound of $a$) and
		(for every real number $L$ such that $L$ is lower bound of a $L \leq K$).
	\end{definition}
	
	\begin{lemma}[MonIncCon]
		Let $a$ be a monotonically increasing bounded sequence. Then $a$ converges.
	\end{lemma}
	
	\begin{proof}
		For every $n$ $a[n] \leq LeastUpper(a)$ (by UpperBoundSeq, LeastUpperBoundSeq).

		Let us show that for every positive real number $\epsilon$ there exists $N$ such that $(LeastUpper(a) - \epsilon) < a[N]$.
		
		\begin{subproof}
			Assume the contrary.
			
			Take a positive real number $\epsilon$ such that for every $N$ not($(LeastUpper(a) - \epsilon) < a[N]$).
			
			Let us show that for every $n$ $a[n] \leq (LeastUpper(a) - \epsilon)$.
			
			\begin{subproof}
				Let $n$ be a natural number.
				
				We have not($(LeastUpper(a) - \epsilon) < a[n]$).
				
				Therefore $(LeastUpper(a) - \epsilon) \geq a[n]$ (by NotRuleOrder).
				
				Hence $a[n] \leq (LeastUpper(a) - \epsilon)$.
				
			\end{subproof}
			
			Hence $(LeastUpper(a) - \epsilon)$ is upper bound of $a$ (by UpperBoundSeq).
			
			$LeastUpper(a) - (LeastUpper(a) - \epsilon) .= LeastUpper(a) + (-LeastUpper(a) + \epsilon)$ (by MinusRule1, MinusRule2)
			
			$.= (LeastUpper(a) - LeastUpper(a)) + \epsilon$ (by AssAdd)
			
			$.= 0 + \epsilon$ (by Neg)
			
			$.= \epsilon + 0$ (by ComAdd)
			
			$.= \epsilon$ (by Zero).
			
			Hence $(LeastUpper(a) - \epsilon) < LeastUpper(a)$.
			
			Hence not($(LeastUpper(a) - \epsilon) \geq LeastUpper(a)) (by NotRuleOrder)$.
			
			Contradiction (by LeastUpperBoundSeq).
			
		\end{subproof}
		
		Let us show that $a$ converges to $LeastUpper(a)$.
		
		\begin{subproof}
			Let $\epsilon$ be a positive real number.
			
			Take $N$ such that $(LeastUpper(a) - \epsilon) < a[N]$.
			
			Let us show that for every $n$ such that $N < n$ $dist(a[n],LeastUpper(a)) < \epsilon$.
			
			\begin{subproof}
				Assume $N < n$.
				
				Hence $a[N] \leq a[n]$ (by MonInc).
				
				We have $a[n] \leq LeastUpper(a)$.
				
				Hence $dist(a[n],LeastUpper(a)) = abs(LeastUpper(a) - a[n]) = LeastUpper(a) - a[n]$.
				
				We have $(LeastUpper(a) - \epsilon) + \epsilon < a[N] + \epsilon$ (by MixedAddInvariance).
				
				We have $((LeastUpper(a) - \epsilon) + \epsilon) - a[N] < (a[N] + \epsilon) - a[N]$ (by MixedAddInvariance).
				
				$((LeastUpper(a) - \epsilon) + \epsilon) - a[N] .= (LeastUpper(a) + (-\epsilon + \epsilon)) - a[N]$ (by AssAdd)
				
				$.= (LeastUpper(a) + (\epsilon - \epsilon)) - a[N]$ (by ComAdd)
				
				$.= (LeastUpper(a) + 0) - a[N]$ (by Neg)
				
				$.= LeastUpper(a) - a[N]$ (by Zero).
				
				$(a[N] + \epsilon) - a[N] .= (\epsilon + a[N]) - a[N]$ (by ComAdd)
				
				$.= \epsilon + (a[N] - a[N])$ (by AssAdd)
				
				$.= \epsilon + 0$ (by Neg)
				
				$.= \epsilon$ (by Zero).
				
				Hence $LeastUpper(a) - a[N] < \epsilon$.
				
				We have $LeastUpper(a) - a[n] \leq LeastUpper(a) - a[N]$.
				
				Hence $dist(a[n],LeastUpper(a)) < \epsilon$ (by MixedTransitivity).
				
			\end{subproof}
			
		\end{subproof}
	\end{proof}
	
	\begin{theorem}[MonCon]
		Let $a$ be a monotonic sequence. $a$ converges iff $a$ is bounded.
	\end{theorem}
	
	\begin{proof}
		We have (If $a$ converges then $a$ is bounded) (by ConvergentImpBounded).
		
		Assume $a$ is bounded.
		
		Case $a$ is monotonically increasing. Then $a$ converges (by MonIncCon). 
		end.
		
		Case $a$ is monotonically decreasing.
		
		\begin{subproof}
			Let us show that $(-1) \cdottwo a$ is monotonically increasing.
			
			\begin{subproof}
				Assume $n \leq m$.
				
				Then $a[n] \geq a[m]$ (by MonDec).
				
				Then $-a[n] \leq -a[m]$ (by OrdMirrorLeq).
				
				$((-1) \cdottwo a)[n] .= (-1) \cdot a[n]
				.= -a[n]$ (by MinusRule4).
				
				$((-1) \cdottwo a)[m] .= (-1) \cdot a[m]
				.= -a[m]$ (by MinusRule4).
				
				Hence $((-1) \cdottwo a)[n] \leq ((-1) \cdottwo a)[m]$.
				
			\end{subproof}
			
			Let us show that $(-1) \cdottwo a$ is bounded.
			
			\begin{subproof}
				Take a real number $K$ such that for every $n$ $abs(a[n]) \leq K$ (by BoundedSequence).
				
				Let us show that for every $n$ $abs(((-1) \cdottwo a)[n]) \leq K$.
				
				\begin{subproof}
					$abs(((-1) \cdottwo a)[n]) .= abs((-1) \cdot a[n])$ (by SequenceConstProd)
					
					$.= abs(-a[n])$ (by MinusRule4)
					
					$.= abs(a[n])$ (by AbsPosNeg).
					
					Hence $abs(((-1) \cdottwo a)[n]) \leq K$.
					
				\end{subproof}
				
				Hence $(-1) \cdottwo a$ is bounded by $K$ (by BoundedBy).
				
			\end{subproof}
			
			Hence $(-1) \cdottwo a$ converges (by MonIncCon).
			
			Take a real number $x$ such that $(-1) \cdottwo a$ converges to $x$ (by Conv).
			
			Let us show that $(-1) \cdottwo ((-1) \cdottwo a) = a$.
			
			\begin{subproof}
				Let us show that for every $n$ $((-1) \cdottwo ((-1) \cdottwo a))[n] = a[n]$.
				
				\begin{subproof}
					Let $n$ be a natural number.
					
					$((-1) \cdottwo ((-1) \cdottwo a))[n] .= (-1) \cdot ((-1) \cdottwo a)[n]$ (by SequenceConstProd)
					
					$.= (-1) \cdot ((-1) \cdot a[n])$ (by SequenceConstProd)
					
					$.= -(-a[n])$ (by MinusRule4)
					
					$.= a[n]$ (by MinusRule2).
										
				\end{subproof}
				
				Hence $(-1) \cdottwo ((-1) \cdottwo a) = a$ (by SequenceEq).
				
			\end{subproof}
			
			Then $(-1) \cdottwo ((-1) \cdottwo a)$ converges to $(-1) \cdot x$ (by ProdConstConv).
			
			Hence $a$ converges (by Conv).
			
		\end{subproof}
		
	\end{proof}


\end{forthel}

\subsection{helper.ftl}

\begin{forthel}
	\noindent [read Sequences/Naturals.ftl]\\
	Let $n,m$ denote natural numbers.\\
	Let $a,b,c,d$ denote real numbers.
	
	
	
	\begin{lemma}[MaxIneqDummy]
	Let $a,b$ be real numbers. Then $b \leq max(a,b)$.
	\end{lemma}
	
	\begin{lemma}[NotRuleOrder]
	$a < b$ iff $not(a \geq b)$.
	\end{lemma}
	
	\begin{lemma}[MixedAddInvariance]
	$a < c \wedge b \leq d \Rightarrow a + b < c + d$.
	\end{lemma}
	
	\begin{lemma}[TwoHalf]
	$(inv(2) \cdot a) + (inv(2) \cdot a) = a$.
	\end{lemma}
	
	\begin{lemma}[MinusRule5]
	Let $a,b$ be real numbers. 
	Then $(a \cdot (-b)) = -(a \cdot b) and ((-b) \cdot a) = -(b \cdot a)$.
	\end{lemma}
	\begin{proof}
	(1) $a \cdot (-b) .= a \cdot ((-1) \cdot b)$ (by MinusRule4)
	$.= (a \cdot (-1)) \cdot b$ (by AssMult)
	$.= ((-1) \cdot a) \cdot b$ (by ComMult)
	$.= (-1) \cdot (a \cdot b)$ (by AssMult)
	$.= -(a \cdot b)$ (by MinusRule4).
	
	$((-b) \cdot a) .= -(b \cdot a)$ (by ComMult, 1).
	\end{proof}
	
	\begin{lemma}[MinusRule6]
	Let $a,b$ be real numbers. 
	Then $((-a) \cdot (-b)) = a \cdot b$.
	\end{lemma}
	\begin{proof}
	$((-a) \cdot (-b)) .= -(a \cdot (-b))$ (by MinusRule5)
	$.= -(-(a \cdot b))$ (by MinusRule5)
	$.= a \cdot b$ (by MinusRule2).
	\end{proof}
	
	\begin{lemma}[Binomi1]
	Let $a,b,c,d$ be real numbers.
	Then $(a + b) \cdot (c + d) = ((a \cdot c) + (b \cdot c)) + ((a \cdot d) + (b \cdot d))$.
	\end{lemma}
	\begin{proof}
	$(a + b) \cdot (c + d) .= ((a + b) \cdot c) + ((a + b) \cdot d)$ (by Distrib)
	$.= ((a \cdot c) + (b \cdot c)) + ((a \cdot d) + (b \cdot d))$ (by DistribDummy).
	\end{proof}
	
	\begin{lemma}[Binomi2]
	Let $a,b,c,d$ be real numbers.
	Then $(a - b) \cdot (c - d) = ((a \cdot c) - (b \cdot c)) + (-(a \cdot d) + (b \cdot d))$.
	\end{lemma}
	\begin{proof}
	$(a - b) \cdot (c - d) .= ((a \cdot c) + ((-b) \cdot c)) + ((a \cdot (-d)) + ((-b) \cdot (-d)))$ (by Binomi1)
	$.= ((a \cdot c) - (b \cdot c)) + (-(a \cdot d) + (b \cdot d))$ (by MinusRule5, MinusRule6).
	\end{proof}
	
	
	
	\begin{lemma}[Identity1]
	Let $a,b,c,d$ be real numbers. 
	Then $(a \cdot b) - (c \cdot d) = ((a - c) \cdot (b - d)) + ((c \cdot (b - d)) + (d \cdot (a - c)))$.
	\end{lemma}
	\begin{proof}
	$((a - c) \cdot (b - d)) + ((c \cdot (b - d)) + (d \cdot (a - c))) 
	.= (((a \cdot b) - (c \cdot b)) + (-(a \cdot d) + (c \cdot d))) + ((c \cdot (b - d)) + (d \cdot (a - c)))$ (by Binomi2)
	$.= (((a \cdot b) - (c \cdot b)) + (-(a \cdot d) + (c \cdot d))) + (((c \cdot b) + (c \cdot (-d))) + ((d \cdot a) + (d \cdot (-c))))$ (by Distrib)
	$.= (((a \cdot b) - (c \cdot b)) + (-(a \cdot d) + (c \cdot d))) + (((c \cdot b) + (-(c \cdot d))) + ((d \cdot a) + (-(d \cdot c))))$ (by MinusRule5)
	$.= ((c \cdot b) + (-(c \cdot d))) + ((((a \cdot b) - (c \cdot b)) + (-(a \cdot d) + (c \cdot d))) + ((d \cdot a) + (-(d \cdot c))))$ (by BubbleAdd)
	$.= ((c \cdot b) + (-(c \cdot d))) + (((a \cdot b) - (c \cdot b)) + ((-(a \cdot d) + (c \cdot d)) + ((a \cdot d) + (-(c \cdot d)))))$ (by AssAdd, ComMult)
	$.= ((c \cdot b) + (-(c \cdot d))) + (((a \cdot b) - (c \cdot b)) + ((-(a \cdot d) + (c \cdot d)) + (-(-((a \cdot d) + (-(c \cdot d)))))))$ (by MinusRule2)
	$.= ((c \cdot b) + (-(c \cdot d))) + (((a \cdot b) - (c \cdot b)) + ((-(a \cdot d) + (c \cdot d)) + (-(-(a \cdot d) + (c \cdot d)))))$ (by ComAdd, MinusRule3)
	$.= ((c \cdot b) + (-(c \cdot d))) + (((a \cdot b) - (c \cdot b)) + 0)$ (by Neg)
	$.= ((-(c \cdot d)) + (c \cdot b)) + (-(c \cdot b) + (a \cdot b))$ (by ComAdd, Zero)
	$.= (-(c \cdot d)) + ((c \cdot b) + (-(c \cdot b) + (a \cdot b)))$ (by AssAdd)
	$.= (-(c \cdot d)) + (((c \cdot b) -(c \cdot b)) + (a \cdot b))$ (by AssAdd)
	$.= -(c \cdot d) + (0 + (a \cdot b))$ (by Neg)
	$.= -(c \cdot d) + ((a \cdot b) + 0)$ (by ComAdd)
	$.= -(c \cdot d) + (a \cdot b)$ (by Zero)
	$.= (a \cdot b) - (c \cdot d)$ (by ComAdd).
	\end{proof}
	
	
	
	\begin{signature}[Sqrt]
	Let $x$ be a positive real number. $sqrt(x)$ is a positive real number such that $sqrt(x) \cdot sqrt(x) = x$.
	\end{signature}
	
	\begin{lemma}[AbsTriangleIneq2]
	Let $x,y$ be real numbers. Then $abs(x) - abs(y) \leq abs(x - y)$.
	\end{lemma}
	\begin{proof}
	$abs(x) .= abs(x + ((-y) + y))$ (by Zero, Neg, ComAdd)
	$.= abs((x + (-y)) + y)$ (by AssAdd).
	$abs((x + (-y)) + y) \leq abs(x - y) + abs(y)$ (by AbsTriangleIneq).
	Hence $abs(x) \leq abs(x + (-y)) + abs(y)$.
	$abs(x) + (-abs(y)) \leq (abs(x - y) + abs(y)) + (-abs(y))$ (by LeqAddInvariance).
	$(abs(x - y) + abs(y)) + (-abs(y)) = abs(x - y)$ (by AssAdd, Neg, Zero).
	Hence $abs(x) - abs(y) \leq abs(x - y)$ (by LeqTransitivity).
	\end{proof}
	
	
	\begin{lemma}[InvAdd]
	Let $a,b,c,d$ be real numbers. Assume $(a \neq 0 and b \neq 0) and (c \neq 0 and d \neq 0)$. 
	Then $(a \cdot inv(b)) + (c \cdot inv(d)) = ((a \cdot d) + (b \cdot c)) \cdot inv(b \cdot d)$.
	\end{lemma}
	\begin{proof}
	$(a \cdot inv(b)) + (c \cdot inv(d)) .= ((a \cdot inv(b)) \cdot 1) + (1 \cdot (c \cdot inv(d)))$ (by One, OneDummy)
	$.= ((a \cdot inv(b)) \cdot (d \cdot inv(d))) + ((b \cdot inv(b)) \cdot (c \cdot inv(d)))$ (by Inverse)
	$.= (a \cdot (inv(b) \cdot (d \cdot inv(d)))) + (b \cdot (inv(b) \cdot (c \cdot inv(d))))$ (by AssMult)
	$.= (a \cdot ((inv(b) \cdot d) \cdot inv(d))) + (b \cdot ((inv(b) \cdot c) \cdot inv(d)))$ (by AssMult)
	$.= (a \cdot ((d \cdot inv(b)) \cdot inv(d))) + (b \cdot ((c \cdot inv(b)) \cdot inv(d)))$ (by ComMult)
	$.= (a \cdot (d \cdot (inv(b) \cdot inv(d)))) + (b \cdot (c \cdot (inv(b) \cdot inv(d))))$ (by AssMult)
	$.= ((a \cdot d) \cdot (inv(b) \cdot inv(d))) + ((b \cdot c) \cdot (inv(b) \cdot inv(d)))$ (by AssMult)
	$.= ((a \cdot d) \cdot inv(b \cdot d)) + ((b \cdot c) \cdot inv(b \cdot d)) $(by InvRule2)
	$.= ((a \cdot d) + (b \cdot c)) \cdot inv(b \cdot d)$ (by DistribDummy).
	\end{proof}
	
	
	
	\begin{lemma}[InvCanc]
	Let $a, b$ be real numbers. Assume $a \neq 0$ and $b \neq 0$.
	Then $(a \cdot inv(b)) \cdot (b \cdot inv(a)) = 1$.
	\end{lemma}
	\begin{proof}
	$(a \cdot inv(b)) \cdot (b \cdot inv(a)) .= ((a \cdot inv(b)) \cdot b) \cdot inv(a)$ (by AssMult)
	$.= (a \cdot (inv(b) \cdot b)) \cdot inv(a)$ (by AssMult)
	$.= (a \cdot 1) \cdot inv(a)$ (by InvDummy)
	$.= a \cdot inv(a)$ (by One)
	$.= 1$ (by Inverse).
	\end{proof}
	
	
	\begin{lemma}[NegMultInvariance]
	Let $x, y, z$ be real numbers.
	$(x < y \wedge z < 0) \Rightarrow z \cdot x > z \cdot y$.
	\end{lemma}
	\begin{proof}
	Assume $x < y \wedge z < 0$.
	Therefore $pos(-z)$.    
	Hence $(-z) \cdot x < (-z) \cdot y$.
	Hence $-((-z) \cdot x) > -((-z) \cdot y)$ (by OrdMirror).
	$-((-z) \cdot x) .= (-(-z)) \cdot x$ (by MinusRule5)
	$.= z \cdot x$ (by MinusRule2).
	$-((-z) \cdot y) .= (-(-z)) \cdot y$ (by MinusRule5)
	$.= z \cdot y$ (by MinusRule2).
	\end{proof}
	
	
	\begin{lemma}[InvSwapIneq]
	Let $a, b, c, d$ be positive real numbers. Assume $(a \neq 0$ and $b \neq 0)$ and $(c \neq 0 and d \neq 0)$. 
	If $a \cdot inv(b) < c \cdot inv(d)$ then $b \cdot inv(a) > d \cdot inv(c)$.
	\end{lemma}
	\begin{proof}
	Assume $a \cdot inv(b) < c \cdot inv(d)$.
	We have $pos(b \cdot inv(a))$ and $pos(d \cdot inv(c))$.
	$(b \cdot inv(a)) \cdot (a \cdot inv(b)) < (b \cdot inv(a)) \cdot (c \cdot inv(d))$ (by MultInvariance).
	$(b \cdot inv(a)) \cdot (a \cdot inv(b)) = $1 (by InvCanc).
	Hence $1 < (b \cdot inv(a)) \cdot (c \cdot inv(d))$.
	$(d \cdot inv(c)) \cdot 1 < (d \cdot inv(c)) \cdot ((b \cdot inv(a)) \cdot (c \cdot inv(d)))$ (by MultInvariance).
	$d \cdot inv(c) < (d \cdot inv(c)) \cdot ((b \cdot inv(a)) \cdot (c \cdot inv(d)))$ (by One).
	$(d \cdot inv(c)) \cdot ((b \cdot inv(a)) \cdot (c \cdot inv(d))) .= ((b \cdot inv(a)) \cdot (c \cdot inv(d))) \cdot (d \cdot inv(c))$ (by ComMult)
	$.= (b \cdot inv(a)) \cdot ((c \cdot inv(d)) \cdot (d \cdot inv(c)))$ (by AssMult)
	$.= (b \cdot inv(a)) \cdot 1$ (by InvCanc)
	$.= b \cdot inv(a)$ (by One).   
	\end{proof}
	
	
	
	\begin{lemma}[AbsInv]
	Let $x$ be a real number. Assume $x \neq 0$. Then $abs(inv(x)) = inv(abs(x))$.
	\end{lemma}
	\begin{proof}
	We have $pos(abs(inv(x)))$ and $pos(inv(abs(x)))$.
	$pos(1)$ (by OnePos).
	(1) Hence $( abs(inv(x)) = abs(abs(inv(x)))$ and $inv(abs(x)) = abs(inv(abs(x))) )$ and $abs(1) = 1$ (by AbsValue).
	$abs(inv(x)) .= abs(abs(inv(x)))$ (by 1)
	$.= abs(abs(inv(x)) \cdot 1)$ (by One)
	$.= abs(abs(inv(x)) \cdot (abs(x) \cdot inv(abs(x))))$ (by Inverse)
	$.= abs((abs(inv(x)) \cdot abs(x)) \cdot inv(abs(x)))$ (by AssMult)
	$.= abs(abs(inv(x) \cdot x) \cdot inv(abs(x)))$ (by AbsMult)
	$.= abs(abs(1) \cdot inv(abs(x)))$ (by InvDummy)
	$.= abs(1 \cdot inv(abs(x)))$ (by 1) 
	$.= abs(inv(abs(x)))$ (by OneDummy)
	$.= inv(abs(x))$ (by 1).    
	\end{proof}
	
	
	\begin{lemma}[InvOne]
	$inv(1) = 1$.
	\end{lemma}
	\begin{proof}
	$1 .= 1 \cdot inv(1)$ (by Inverse)
	$.= inv(1)$ (by OneDummy).
	\end{proof}   
	
\end{forthel}


\section{Remarks}
\subsection{Structures}


\subsection{Text Comparisons}



\section{{From \LaTeX} to ForTheL}



\section{Discussion}


\begin{thebibliography}{1}

\bibitem{Rudin}
  Walter Rudin,
  \textit{Principles of mathematical analysis},
  McGraw-Hill,
  1976.

\end{thebibliography}
  


\end{document}
