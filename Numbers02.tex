\documentclass{article}
\usepackage[english]{babel}
\usepackage{enumerate, latexsym, amssymb, amsmath}
\usepackage{framed, multicol}
\newenvironment{forthel}{\begin{leftbar}}{\end{leftbar}}

%%%%%%%%%% Start TeXmacs macros
\newcommand{\tmaffiliation}[1]{\\ #1}
\newcommand{\tmem}[1]{{\em #1\/}}
\newenvironment{enumeratenumeric}{\begin{enumerate}[1.] }{\end{enumerate}}
\newenvironment{proof}{\noindent\textbf{Proof\ }}{\hspace*{\fill}$\Box$\medskip}
\newenvironment{quoteenv}{\begin{quote} }{\end{quote}}
\newtheorem{axiom}{Axiom}
\newtheorem{lemma}{Lemma}
\newtheorem{theorem}{Theorem}
\newtheorem{definition}{Definition}
\newtheorem{signature}{Signature}
\newtheorem{proposition}{Proposition}
%%%%%%%%%% End TeXmacs macros

\newcommand{\event}{UITP 2018}
\newcommand{\dom}{Dom}
\newcommand{\fun}{aFunction}
\newcommand{\sym}{sym}
\newcommand{\halfline}{{\vspace{3pt}}}
\newcommand{\tab}{{\hspace{1cm}}}
\newcommand{\ball}[2]{B_{#1}(#2)}
\newcommand{\llbracket}{[}
\newcommand{\rrbracket}{]}
\newcommand{\less}[1]{<_{#1}}
\newcommand{\greater}[1]{>_{#1}}
\newcommand{\leeq}[1]{{\leq}_{#1}}
\newcommand{\supr}[1]{\mathrm{sup}_{#1}}
\newcommand{\RR}{\mathbb{R}}
\newcommand{\QQ}{\mathbb{Q}}
\newcommand{\ZZ}{\mathbb{Z}}
\newcommand{\NN}{\mathbb{N}}

\begin{document}

\title{An SAD3 Formalization of the Basic Number Systems for Walter Rudin's
\it{Principles of Mathematical Analysis}}

\author{Peter Koepke}

\date{August 25, 2018}

\maketitle


\section{Introduction}

The SAD3 proof-checking system checks the logical correctness of texts written in an input language which is acceptable and readable as common mathematical language. Texts should resemble the style of undergraduate textbooks. To test the viability of this approach we are currently formalizing some initial chapters of the {\it Principles of Mathematical Analysis} by Walter Rudin \cite{Rudin}. SAD3 accepts texts in a native ForTheL (Formula Theory Language) format. We are also testing a {\LaTeX} add-on which accepts texts in {\LaTeX} format and transforms them into ForTheL.

The present formalization deals with pages 1 to 9 of \cite{Rudin}. Like in the book, the ordered field $\mathbb{R}$ of real numbers is postulated. We construe the structures of integer and rational numbers as substructures of $\mathbb{R}$:
$$\NN \subseteq \ZZ \subseteq \QQ \subseteq \RR.$$
This has the advantage that the real addition and multiplication can be used for those substructures and we do not have to talk about ordered sets and fields in general. A large part of the material of Rudin appears in the formalization in some form. We have turned the axioms about general linear orders and fields into axioms for the real numbers.



We first did the formalization in a ForTheL file {\tt Numbers03.ftl}. After this file was accepted by SAD3 it was rewritten to a {\LaTeX} file
{\tt Numbers02.tex}. The {\LaTeX} file may contain further text that is typeset but not checked. After typesetting the ForTheL content is indicated by a vertical line.  In this way this whole document is a single SAD3 document, whose ForTheL content is proofchecked. Towards the end of the formalization we put the original Rudin text side-by-side with the formalization for comparison.



\section{The Formalization}
\subsection{Some Set-Theoretic Terminology}

Rudin introduces this on page 3.

\begin{forthel}
Let $A,B$ stand for sets.
Let $x \in A$ denote $x$ is an element of $A$.
Let $x$ is \emph{in} $A$ denote $x$ is an element of $A$.
Let $x \notin A$ denote $x$ is not an element of $A$.

\begin{signature} The \emph{empty set} is the set that has no elements.
Let $\emptyset$ denote the empty set.
\end{signature}

\begin{definition} $A$ is \emph{nonempty} iff $A$ has an element.
\end{definition}

\begin{definition} A subset of $B$ is a set $A$ such that every element 
of $A$ is an element of $B$. 
Let $A \subseteq B$ stand for $A$ is a subset of $B$.
Let $B \supseteq A$ stand for $A$ is a subset of $B$.
\end{definition}

\begin{definition} A \emph{proper subset} of $B$ is a subset $A$ of $B$ such that there is an element of $B$ that is not in $A$.
\end{definition}

\begin{proposition} $A \subseteq A$. \end{proposition}

\begin{proposition} If $A \subseteq B$ and $B \subseteq A$ then $A = B$. \end{proposition}


\begin{definition} $A \cup B = \{x \mid x \in A \text{ or } x \in B\}$. \end{definition}

\end{forthel}

\subsection{The Field of Real Numbers}

Rudin introduces $\RR$ in Theorem 1.19 and refers to the ordered field 
axioms in 1.5, 1.6, 1.12. Our propositions correspond to Rudin's 1.14 - 1.16 on fields. 

\begin{forthel}
[number/-s]
\begin{signature} A \emph{real number} is a notion.\end{signature}
\begin{signature} $\RR$ is the set of \emph{real numbers}.
Let $x,y,z$ denote real numbers.
\end{signature}

\begin{signature} $x + y$ is a real number.
Let the \emph{sum} of $x$ and $y$ denote $x + y$.\end{signature}

\begin{signature} $x \cdot y$ is a real number.
Let the \emph{product} of $x$ and $y$ denote $x \cdot y$.\end{signature}

\begin{signature} $x < y$ is an atom.
Let $x > y$ stand for $y < x$.
Let $x \leq y$ stand for $x < y \vee x = y$.
Let $x \geq y$ stand for $y \leq x$.\end{signature}

\begin{axiom} $x < y \wedge x \neq y \wedge \neg y < x$
or $\neg x < y \wedge x = y \wedge \neg y < x$
or $\neg x < y \wedge x \neq y \wedge y < x$.\end{axiom}

\begin{axiom} If $x < y$ and $y < z$ then $x < z$.
\end{axiom}

\begin{proposition}[15b] $x \leq y$ iff not $x > y$.\end{proposition}

\begin{axiom} $x + y = y + x$.\end{axiom}

\begin{axiom} $(x + y) + z = x + (y + z)$.\end{axiom}

\begin{signature} $0$ is a real number such that
for every real number $x$ $x + 0 = x$.\end{signature}

\begin{signature} $-x$ is a real number such that $x + (-x) = 0$.
\end{signature}

\begin{axiom} $x \cdot y = y \cdot x$.\end{axiom}

\begin{axiom} $(x \cdot y) \cdot z = x \cdot (y \cdot z)$.
\end{axiom}

\begin{signature} $1$ is a real number such that $1 \neq 0$ and 
for every real number $x$ $1 \cdot x = x$.\end{signature}

\begin{signature} Assume $x \neq 0$. $1/x$ is a real number
such that $x \cdot (1/x) = 1$.\end{signature}

\begin{axiom}[Distributivity] $x \cdot (y + z) = (x \cdot y) + (x \cdot z)$.
\end{axiom}
\begin{proposition} $(y \cdot x) + (z \cdot x) = (y + z) \cdot x$.
\end{proposition}
\begin{proof}
$$(y \cdot x) + (z \cdot x) .= 
(x \cdot y) + (x \cdot z) .=
x \cdot (y + z).=
(y + z) \cdot x$$.
\end{proof}

\begin{proposition} If $x + y = x + z$ then $y = z$.
\end{proposition}
\begin{proof} Assume $x + y = x + z$. Then 
$$y = (-x+x) + y = -x + (x+y) = -x + (x+z) = (-x+x) + z = z.$$
\end{proof}

\begin{proposition} If $x + y = x$ then $y = 0$.\end{proposition}
\begin{proposition} If $x + y = 0$ then $y = -x$.\end{proposition}
\begin{proposition} $-(-x) = x$.\end{proposition}


\begin{proposition} If $x \neq 0$ and $x \cdot y = x \cdot z$ 
then $y = z$. \end{proposition}
\begin{proof} Let $x \neq 0$ and $x \cdot y = x \cdot z$.
$$y = 1 \cdot y = ((1/x) \cdot x) \cdot y = (1/x) \cdot (x \cdot y) =
(1/x) \cdot (x \cdot z) = ((1/x) \cdot x) \cdot z = 1 \cdot z = z.$$
\end{proof}

\begin{proposition} If $x \neq 0$ and $x \cdot y = x$ then $y = 1$.
\end{proposition}
\begin{proposition} If $x \neq 0$ and $x \cdot y = 1$ then $y = 1/x$.
\end{proposition}

\begin{proposition} If $x \neq 0$ then $1/(1/x) = x$.
\end{proposition}
\begin{proposition} $0 \cdot x = 0$. \end{proposition}
\begin{proposition} If $x \neq 0$ and $y \neq 0$ then $x \cdot y \neq 0$.
\end{proposition}
\begin{proposition} $(-x) \cdot y = -(x \cdot y)$. \end{proposition}
\begin{proof} $(x \cdot y) + (-x \cdot y) = (x + (-x)) \cdot y 
= 0 \cdot y = 0$.
\end{proof}

\begin{proposition} $-x = -1 \cdot x$. \end{proposition}

\begin{proposition}$(-x) \cdot (-y) = x \cdot y$.
\end{proposition}
\begin{proof} $(-x)\cdot (-y)=-(x\cdot(-y))=-((-y)\cdot x)=
-(-(y\cdot x))=y\cdot x=x\cdot y$. 
\end{proof}

Let $x - y$ stand for $x + (-y)$.
Let $\frac{x}{y}$ stand for $x \cdot (1/y)$.

\end{forthel}

\subsection{The Ordered Real Field}
Rudin introduces ordered fields in 1.17, corresponding to the subsequent two axioms. Our propositions formalize Rudin's 1.18.

\begin{forthel}

\begin{axiom} If $y < z$ then $x + y < x + z$ and $y + x < z + x$.
\end{axiom}

\begin{axiom} If $x > 0$ and $y > 0$ then $x \cdot y > 0$.
\end{axiom}

\begin{definition} $x$ is \emph{positive} iff $x > 0$.
\end{definition}

\begin{definition} $x$ is \emph{negative} iff $x < 0$.
\end{definition}


\begin{proposition} $x > 0$ iff $-x < 0$. \end{proposition}

\begin{proposition} If $x > 0$ and $y < z$ then $x \cdot y < x \cdot z$.
\end{proposition}
\begin{proof} Let $x > 0$ and $y < z$.
$z - y > y - y = 0$.
$x \cdot (z - y) > 0$.
$x \cdot z = (x \cdot (z - y)) + (x \cdot y)$.
$(x \cdot (z - y)) + (x \cdot y)  > 0 + (x \cdot y)$.
$0 + (x \cdot y) = x \cdot y$.
\end{proof}

\begin{proposition} If $x > 0$ and $y < z$ then 
$y \cdot x < z \cdot x$.
\end{proposition}

\begin{proposition} If $x \neq 0$ then $x \cdot x > 0$.
\end{proposition}


\begin{proposition} $1 > 0$.\end{proposition}


\begin{proposition} $x < y$ iff $-x > -y$.\end{proposition}
\begin{proof}
$x < y \Leftrightarrow x - y < 0$. 
$x - y < 0 \Leftrightarrow (-y) + x < 0$. 
$(-y) + x < 0 \Leftrightarrow (-y)+(-(-x)) < 0$.
$(-y)+(-(-x)) < 0 \Leftrightarrow (-y)-(-x) < 0$.
$(-y)-(-x) < 0 \Leftrightarrow -y < -x$.
\end{proof}

\begin{proposition} If $x < 0$ and $y < z$ then 
$x \cdot y > x \cdot z$.\end{proposition}
\begin{proof} Let $x < 0$ and $y < z$.
$-x > 0$.
$(-x)\cdot y < (-x)\cdot z$.
$-(x\cdot y) < -(x\cdot z)$.
\end{proof}

\begin{proposition} If $x < 0$ and $y < z$ then 
$y \cdot x > z \cdot x$.\end{proposition}

\begin{proposition}[Successor] $x + 1 > x$.\end{proposition}
\begin{proposition} $x - 1 < x$. \end{proposition}


\begin{proposition} If $0 < x$ then $0 < 1/x$.
\end{proposition}

\begin{proposition} If $0 < x < y$ then $1/y < 1/x$.
\end{proposition}
\begin{proof}
Assume $0 < x < y$.
Then $1/x < 1/y$ or $1/x = 1/y$ or $1/y < 1/x$.
Case $1/x < 1/y$. Then
$$1 = x \cdot (1/x) = (1/x) \cdot x < (1/x) \cdot y = 
y \cdot (1/x) < y \cdot (1/y) = 1.$$ 
Contradiction. end.

Case $1/x = 1/y$. Then
$$1 = x \cdot (1/x) < y \cdot (1/y) = 1.$$ 
Contradiction. end.

Case $1/y < 1/x$. end.
\end{proof}

\end{forthel}

\subsection{Upper and Lower Bounds}
This corresponds to Rudin's 1.7. and 1.8. We also study lower bounds and their relation to upper bounds, because lower bounds appear more natural in the context of induction.
\begin{forthel}
\begin{definition} Let $E$ be a subset of $\RR$. 
An \emph{upper bound} of $E$ is a
real number $b$ such that for all elements $x$ of $E$ $x \leq b$.
\end{definition}

\begin{definition} Let $E$ be a subset of $\RR$. $E$ is 
\emph{bounded above} iff
$E$ has an upper bound.\end{definition}

\begin{definition} Let $E$ be a subset of $\RR$. 
A \emph{lower bound} of $E$ is a
real number $b$ such that for all elements $x$ of $E$ $x \geq b$.
\end{definition}

\begin{definition} Let $E$ be a subset of $\RR$. 
$E$ is \emph{bounded below} iff
$E$ has a lower bound.\end{definition}

\begin{definition} Let $E$ be a subset of $\RR$ such that $E$ is bounded above.
A \emph{least upper bound} of $E$ is a real number $a$ such that
$a$ is an upper bound of $E$ and for all $x$ if $x < a$ then $x$ 
is not an upper bound of $E$.\end{definition}

\begin{definition} Let $E$ be a subset of $\RR$ such that $E$ 
is bounded below.
A \emph{greatest lower bound} of $E$ is a real number $a$ such that
$a$ is a lower bound of $E$ and for all $x$ if $x > a$ then $x$ is 
not a lower bound of $E$.\end{definition}

\begin{axiom} Assume that $E$ is a nonempty subset of $\RR$ 
such that $E$ is bounded
above. Then $E$ has a least upper bound.
\end{axiom}

\begin{definition} Let $E$ be a subset of $\RR$. 
$E^- = \{-x \mid x \in E\}$.\end{definition}

\begin{lemma} Let $E$ be a subset of $\RR$.
$x$ is an upper bound of $E$ iff $-x$ is a lower bound of $E^-$.
\end{lemma}

\begin{theorem} Assume that $E$ is a nonempty subset of $\RR$ 
such that $E$ is bounded below.
Then $E$ has a greatest lower bound.\end{theorem}
\begin{proof}
Take a lower bound $a$ of $E$.
$-a$ is an upper bound of $E^-$.
Take a least upper bound $b$ of $E^-$.
Let us show that $-b$ is a greatest lower bound of $E$.
$-b$ is a lower bound of $E$. Let $c$ be a lower bound of $E$. 
Then $-c$ is an upper bound of $E^-$.
end. \end{proof}

\end{forthel}

\subsection{Rational Numbers}
Integer and rational numbers are not axiomatized or constructed in Rudin, but simple assumed. We need the following formalizations to make the text self-contained.

\begin{forthel}

\begin{signature} A \emph{rational number} is a real number.
Let $p,q,r$ stand for rational numbers.\end{signature}

\begin{definition} $\QQ$ is the set of rational numbers.
\end{definition}


\begin{lemma} $\QQ \subseteq \RR$.\end{lemma}

\begin{axiom} $p + q$, $p \cdot q$, $0$, $-p$, $1$ are 
rational numbers.\end{axiom}

\begin{axiom} Assume $p \neq 0$. $1/p$ is a rational number.
\end{axiom}

\end{forthel}

\subsection{Integers}

$\ZZ$ is a discrete subring of $\QQ$.

\begin{forthel}

[integer/-s]

\begin{signature} An \emph{integer} is a rational number.
Let $a,b$ stand for integers.\end{signature}

\begin{definition} $\ZZ$ is the set of integers.\end{definition}


\begin{axiom} $0$, $1$ are integers.
\end{axiom}

\begin{axiom} $a + b$, $a \cdot b$, $-a$ are integers.
\end{axiom}

\begin{axiom} There is no integer $a$ such that $0 < a < 1$.
\end{axiom}

\begin{axiom} There exist $a,b$ such that 
$a \neq 0 \wedge p = \frac{b}{a}$.
\end{axiom}

\begin{theorem}[Archimedes1] $\ZZ$ is not bounded above.
\end{theorem}
\begin{proof} Assume the contrary.
$\ZZ$ is nonempty. Take a least upper bound 
$b$ of $\ZZ$.
Let us show that $b - 1$ is an upper bound of $\ZZ$.
Let $x \in \ZZ$. $x + 1 \in \ZZ$. 
$x + 1 \leq b$.
$x = (x + 1) - 1 \leq b - 1$.
end.
\end{proof}

\begin{theorem}[Archimedes2] There is an integer $a$ 
such that $x \leq a$.\end{theorem}
\begin{proof} $x$ is not an upper bound of $\ZZ$ 
(by Archimedes1).
Take an integer $a$ such that not $a \leq x$.
Then $x \leq a$.
\end{proof}

\end{forthel}

\subsection{Positive Integers}
The natural numbers $1,2,3,..$ are defined as positive integers, to use the terminology from Rudin.
\begin{forthel}


\begin{definition} $\NN$ is the set of positive integers.
Let $m,n$ stand for positive integers.\end{definition}

\begin{definition} $\{x\} = \{real numbers y \mid y = x\}$.
\end{definition}

\begin{lemma} $\ZZ = (\NN^- \cup \{0\}) \cup \NN$.
\end{lemma}

\begin{theorem}[Induction] Assume $A \subseteq \NN$ 
and $1 \in A$ and for all $n \in A$ $n + 1 \in A$.
Then $A = \NN$.\end{theorem}

\begin{proof}
Let us show that every element of $\NN$ is an element of $A$. 
	Let $n \in \NN$.
	Assume the contrary.
	Define $F = \{ j \in \NN \mid j \notin A\}$.
	$F$ is nonempty. $F$ is bounded below.
  Take a greatest lower bound $a$ of $F$.
	Let us show that $a+1$ is a lower bound of $F$.
		Let $x \in F$. $x - 1 \in \ZZ$.
		
		Case $x - 1 < 0$. Then $0 < x < 1$. Contradiction. end.

		Case $x - 1 = 0$. Then $x = 1$ and $1 \notin A$. Contradiction. end.
		
		Case $x - 1 > 0$. Then $x - 1 \in \NN$. $x - 1 \in F$.
		    Indeed if $x -1 \notin F$ then $x-1 \in A$ and $x \in A$. 
			$a \leq x - 1$.
			$a + 1 \leq (x - 1) + 1 = x$.
			end.
	end.
	
	Then $a+1 > a$ (by Successor).
	Contradiction.
end.
\end{proof}

\end{forthel}

\subsection{Archimedian properties}
Finally we have formalized what Rudin assumes for the proof of his 1.20. We can now contrast the formalization with Rudin's original.
\begin{forthel}
\begin{lemma} There is an integer $m$ such that
$$m - 1 \leq z < m.$$ \end{lemma}
\begin{proof} 
Define $X = \{ m \in \ZZ \mid z < m\}$.
Then $z$ is a lower bound of $X$.
Take a greatest lower bound $\alpha$ of $X$.
$\alpha < \alpha + 1$.
Take an element $m$ of $X$ such that not $\alpha + 1 \leq m$.
Indeed $\alpha + 1$ is not a lower bound of $X$.
$m < \alpha + 1$. 
$$m - 1 < (\alpha + 1) - 1 = \alpha.$$
Then $m - 1$ is not an element of $X$. 
We have not $z < m-1$ and $m-1 \leq z$.
\end{proof} 
\end{forthel}
\newpage
The next two theorems correspond to Theorems 1.20(a) and (b) of Rudin. We start with the original introductory text of Rudin and then display the two texts in parallel to illustrate the closeness of the formalization to the original.

\vspace{1cm}

The next theorem could be extracted from this construction [of the real numbers] with very little extra effort. However, we prefer to derive it from Theorem 1.19 since this provides a good illustration of what one can do with the least-upper-bound property.

\begin{multicols}{2}

\begin{forthel}
\begin{theorem}[120a] If $x \in \RR$ and $y \in \RR$ and $x > 0$
then there is a 
positive integer $n$ such that $$n \cdot x > y.$$ \end{theorem}
\begin{proof} 
Define $X = \{ n \cdot x \mid n \text{ is a positive integer}\}$.
Assume the contrary.
Then $y$ is an upper bound of $X$.
Take a least upper bound $\alpha$ of $X$.
$\alpha - x < \alpha$ and $\alpha - x$ is not an upper bound of $X$.
Take an element $z$ of $X$ such that not $z \leq \alpha - x$.
Take a positive integer $m$ such that $z = m \cdot x$.
Then $\alpha - x <  m \cdot x$ (by 15b).

$\alpha = (\alpha -x) + x <$

$(m \cdot x) + x = (m+1) \cdot x.$

$(m+1) \cdot x$ is an element of $X$. 
Contradiction. Indeed $\alpha$ is an upper bound of $X$.
\end{proof}

\end{forthel}

\begin{theorem}
(a) If $x\in \mathbb{R}$, $y\in \mathbb{R}$, and $x>0$, then there is a positive integer $n$ such that
$$n x > y.$$
\end{theorem}
\begin{proof}
Let $A$ be the set of all $n x$, where $n$ runs through the positive integers. If $(a)$ were false, then $y$ would be an upper bound of $A$. But then $A$ has a \emph{least} upper bound in $\mathbb{R}$.
Put $\alpha=\sup A$. Since $x>0$, $\alpha-x<\alpha$, and $\alpha-x$ is not an upper bound of $A$. Hence $\alpha-x<m x$ for some positive integer $m$. But then $\alpha<(m+1) x \in A$, which is impossible, since $\alpha$ is an upper bound of $A$.
\end{proof}

\end{multicols}
\newpage
\begin{multicols}{2}

\begin{forthel}

\begin{theorem}[120b] If $x \in \RR$ and $y \in \RR$ and $x < y$ then there exists a rational number $p$ such that $x < p < y$. \end{theorem}
\begin{proof} 
Assume $x < y$.
We have $y - x > 0$.
Take a positive integer $n$ such that 
$n\cdot (y-x) > 1$ (by 120a).
Take an integer $m$ such that
$m -1 \leq n \cdot x < m$.
Then
$$n \cdot x < m = (m - 1) + 1 $$
$$\leq (n \cdot x) + 1 < (n \cdot x) + (n \cdot (y-x))$$
$$ = n \cdot (x + (y - x)) = n \cdot y.$$
$m \leq (n\cdot x) + 1 < n \cdot y$.
$\frac{m}{n} < \frac{n\cdot y}{n}$. Indeed $m < n\cdot y$ and $1/n > 0$.
Then
$$x = \frac{n\cdot x}{n} < \frac{m}{n} < \frac{n\cdot y}{n} = y.$$
Let $p = \frac{m}{n}$. Then $p \in \QQ$ and $x < p < y$.
\end{proof}

\end{forthel}

\columnbreak

\begin{theorem}
If $x\in\mathbb{R}$, $y\in\mathbb{R}$, and $x<y$, then there exists
a $p\in\mathbb{Q}$ such that $x<p<y$.
\end{theorem}
\begin{proof}
Since $x < y$, we have $y-x>0$, and $(a)$ furnishes a positive integer $n$ such that
$$m (y-x) > 1.$$
Apply $(a)$ again, to obtain positive integers $m_1$ and $m_2$ such that
$m_1 > n x$, $m_2 > -n x$. Then
$$-m_2 <n x < m_1.$$
Hence there is an integer $m$ (with $-m_2 \leq m \leq m_1$) such that
$$m-1 \leq n x < m.$$
If we combine these inequalities, we obtain
$$n x < m \leq 1 + n x < n y.$$
Since $n > 0$, it follows that
$$x < \frac{m}{n} < y.$$
This proves $(b)$, with $p = m/n$.
\end{proof}

\end{multicols}

\section{Remarks}
\subsection{Structures}

Rudin introduces ordered sets and fields axiomatically and then postulates the ordered field of reals. Indeed the reals are constructed from the rationals numbers in an Appendix to chapter 1.

Since ForTheL does not provide general structures and mechanisms to express that a structure satisfies some abstract axioms, I have instead postulated the structure $\mathbb{R}$ of the real numbers right away. The axioms of ordered sets and fields are then only stated for this particular structure.

Instead of building up number systems "from below" we define the sets of rational, integer and positive integer numbers as subsets of $\mathbb{R}$. This has the advantage that we can use the real addition and multiplication also for those number systems and they become substructures without substructure mechanisms.

Future versions of SAD3 should allow for elegant handling of structures. This also involves a liberal use of operator symbols like $+$ in several additive structures, without mentioning the structure, e.g., as a subscript. This "overloading" would require some (Prolog-like?) derivations of notions for variables and terms.

\subsection{Text Comparisons}

We cover most of the material on pages 3 - 9 of Rudin. By the different organisation, the two presentations only align on page 9 for some archimedean properties. We illustrate this alignment in a two-column synoptic view of the formalization and the original. One could go through the text statement by statement and discuss similarities and differences. One could also put in further work to minimize differences. To some degree this could take part on the level of the formalization, but eventually the ForTheL language and its interpretation in SAD3 would have to be extended.
I shall only mention a few points.

\subsubsection{Theorem 5 (= Theorem 1.20(a) in Rudin)} 

The statement of the theorem is nearly identical. There are some notational differences like the omision of the multiplication sign $\cdot$ in the original. In ForTheL, premises like $x \in \RR$, $y \in \RR$, $x > 0$ cannot be chained but have to be combined by a logical "and". This probably leads to some logical overhead, since the single premises are only available after breaking-up the "and"'s.

Future versions of SAD3 should allow more chaining of statements. Also in Definitions and Theorems, there could be chains of definitions or conclusions instead of single statements. One could allow listing and enumeration environments of {\LaTeX}. Then Theorem 1.20 could have been formalized as one Theorem with parts $(a)$ and $(b)$.

Rudin uses the $sup$ operator for {\it least} upper bounds. Instead we only use "some" least upper bound. This could be adjusted in some revision of the formalization.

Human readers are very apt at overseeing term equalities and inequalities. The term rewriting capacities of SAD3 are limited as well as the handling of chains of equalities and inequalities, which makes for longer texts. There is room for improvement in future versions of SAD3.

Argumentative connectives like "since" should be included in extensions of ForTheL. Presently they have to be modelled by the "Indeed" construct or by listing of premises.

\subsubsection{Theorem 7 (= Theorem 1.20(b) in Rudin)}

The original text in my 1976 edition contains the typo "a positive integer $n$ such that
$$m (y-x) > 1.\text{"}$$
This $m$ should have been $n$; SAD3 would have spotted the error immediately.


The phrase {\it there exists a $p \in \QQ$ such that $x < p < y$} cannot entered like that into ForTheL, since the phrase $p \in \QQ$ is not understood as a modifier to the variable $p$ but as a statement of its own. This leads to a syntax error. In ForTheL, the modification is instead expressed by the type-like notion {\it there exists a rational number $p$ ...}. It would be desirable to allow variable modifications in quantifications by infix binary predicates. This would also cover cases like {\it for all $\epsilon > 0$ there exists $\delta > 0$ ...}.

Rudin uses an informal argument ("Hence") to get from
$$-m_2 <n x < m_1$$
to the existence of an integer $m$ (with $-m_2 \leq m \leq m_1$) such that
$$m-1 \leq n x < m.$$
This is intuitively obvious, but axiomatically one has to reduce this to formal properties of finiteness. We therefore have provided the extra Lemma 4 and changed the proof of Theorem 7 accordingly.

\section{{From \LaTeX} to ForTheL}
The {\LaTeX}-input to SAD3 is achieved by a simple-minded translation which transforms {\LaTeX}-text into corresponding ForTheL-text. A lot of type-setting information is simply filtered out and not given to SAD3. In particular the {\$}-symbols are filtered out since in ForTheL there is no real difference between symbolic and textual input. Style information like {\tt \textbackslash emph\{word\}} becomes simply {\tt word}.

{\LaTeX}-environments like the definition, lemma, theorem or proof environments are modified into the corresponding ForTheL top level sections. E.g., the text

{\tt \textbackslash begin\{theorem\}}

{\tt theoremtext}

{\tt \textbackslash end\{theorem\}}

becomes

{\tt Theorem. (the translation of) theoremtext}

Some hacks have been employed. Translating 
{\tt \textbackslash mathbb\{R\}} to {\tt R} is problematic since single letters are interpreted as variables and not as constants. Instead I have introduced a macro {\tt \textbackslash RR} which is typeset as $\mathbb{R}$ and where {\tt \textbackslash RR} is a legitimate ForTheL constant.

Actually only text within Forthel environments of the form

{\tt \textbackslash begin\{forthel\}}

{\tt theoremtext}

{\tt \textbackslash end\{forthel\}}

is transmitted to SAD3. Everything outside those environments is filtered away so that one can write free {\LaTeX} outside the ForTheL content. Actually this document is a {\LaTeX}-document containing several ForTheL-environments. These are typeset with a vertical marginal line.

All this filtering is coded into the Haskell file
{\tt LaTeX.hs} which exports the translation function
{\tt fromLaTeX :: String -> String}.
This function is employed in the source code of SAD3 in the file
{\tt Reader.hs} where {\tt fromLaTeX} is applied to the input string within the {\tt reader}-function.

Eventually a simple {\LaTeX}-format should become the standard input format for SAD3. This could be integrated into the standard workflow of mathematicians. If parts of a text should be checked by SAD3 put them into ForTheL-environments and feed the text to SAD3.


\section{Discussion}

Formalizing in SAD3 is not straightforward. The ForTheL language is a relatively rich language which appears like a natural language with a rather complicated (formal) grammar. This makes it at first hard to predict which constructs will be accepted. In the formalization of the proof of Rudin's 1.20(a), I could not name the set of multiples of $x$ by the letter $A$ since this led to an ambiguity, perhaps because $A$ was already pretyped as a set?

Also specific forms of constructs, although logically and intuitively equivalent, may greatly influence checking times or the checkability at all. Probably, some logical reformulations, however simple, may lead to further resolution steps and expand the search space of the ATP. Perhaps some obvious logical equivalences could be handled by simplifications in the reasoner instead of by the ATP?

The user of SAD3 seems to play something like a cooperative trial-and-error game with the ATP. There are cases, where the system is unable to do trivial steps like deducing $A$ from a long conjunction like $(A\wedge B\wedge C\wedge D \wedge E)$ whereas sometimes impressive argumentative chains of lemmas are found immediately. It seems that the heuristics of the ATP may give a "long" formula like $(A\wedge B\wedge C\wedge D \wedge E)$ low priority, whereas the powerful resolution algorithm will find non-trivial lemma applications through unification. It would be desirable to have more intuitions about the abilities of the ATP. Perhaps one could even tune the ATP towards the kind of tasks that SAD3 produces.  

The whole of this document is checked in 3 minutes on a modest HP Compaq laptop. I did not notice a slow-down of the checking process along the length of the document. This supports the expectation that sizable textbook mathematics can be handled by (future versions of) SAD3. To work interactively with the system and get reasonably fast responses it is necessary to intelligently switch on and off the checking algorithms along a text. This will hopefully be supported by an embedding of SAD3 into the Isabelle/jedit editing framework.

Finally it will be a challenge to see whether the formalizations of parts of Chapter 1-4 of Rudin can be linked together or turned into a single readable and checkable proof document.

\begin{thebibliography}{1}

\bibitem{Rudin}
  Walter Rudin,
  \textit{Principles of mathematical analysis},
  McGraw-Hill,
  1976.

\end{thebibliography}
  


\end{document}
